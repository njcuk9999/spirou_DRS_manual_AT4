% ----------------------------------------------------------------------
% formatting constants
% ----------------------------------------------------------------------
% formatting the named variables (from user setup)
\newcommand{\definevariable}[1]{\textcolor{blue}{\{#1\}}}
% formatting the named keywords 
\newcommand{\definekeyword}[1]{\textcolor{red}{\{#1\}}}
% formatting the TODO command
\newcommand{\TODO}[1]{\vspace{0.5cm}\colorbox{black}{\parbox{0.9\textwidth}{\textcolor{green}{!!!! TODO !!!! #1 !!!!}}}\vspace{0.5cm}}

% formatting for the directories trees
\newcommand{\customdirtree}[1]{
\vspace{-0.25cm}
\definecolor{shadecolor}{rgb}{0.85,0.85,0.82}
\begin{shaded}
\renewcommand*\DTstylecomment{\ttfamily\textcolor{black}}
\renewcommand*\DTstyle{\ttfamily\textcolor{blue}}
\dirtree{%
#1}
\end{shaded}
}

\newcommand{\ParamList}[5]{
\vspace{0.5cm}
\noindent\textcolor{red}{\textbf{#1:}}
\begin{itemize}
	\item \textcolor{gray}{\textbf{Default Value:}} #2
	\item \textcolor{gray}{\textbf{Description:}} #3
	\item \textcolor{gray}{\textbf{Used in:}} #4
	\item \textcolor{gray}{\textbf{Defined in:}} #5
\end{itemize}
}

\newcommand{\ParamListCode}[5]{
\vspace{0.5cm}
\noindent\textcolor{red}{\textbf{#1:}}
\begin{itemize}
	\item \textcolor{gray}{\textbf{Default Value:}} 
	\textcolor{blue}{#2}
	\item \textcolor{gray}{\textbf{Description:}} #3
	\item \textcolor{gray}{\textbf{Used in:}} #4
	\item \textcolor{gray}{\textbf{Defined in:}} #5
\end{itemize}
}

\newcommand{\KWList}[5]{
\vspace{0.5cm}
\noindent\textcolor{red}{\textbf{#1:}}
\begin{itemize}
	\item \textcolor{gray}{\textbf{Variable name:}} #2
	\item \textcolor{gray}{\textbf{Description:}} #3
	\item \textcolor{gray}{\textbf{Used in:}} #4
	\item \textcolor{gray}{\textbf{Defined in:}} #5
\end{itemize}
}


% ----------------------------------------------------------------------
% centered hbox
% ----------------------------------------------------------------------
%% better: (general command to vertically center horizontal material)
\newcommand*{\vcenteredhbox}[1]{\begingroup
\setbox0=\hbox{#1}\parbox{\wd0}{\box0}\endgroup}
