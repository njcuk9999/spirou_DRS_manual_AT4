% \iffalse
%%
%% Package `dirtree.dtx'
%% -----------------------------------------------
%% Copyright (C) 2004-2009 Jean-C\^ome Charpentier
%% -----------------------------------------------
%%
%% This work may be distributed and/or modified under the
%% conditions of the LaTeX Project Public License, either version 1.3
%% of this license or (at your option) any later version.
%% The latest version of this license is in
%%   http://www.latex-project.org/lppl.txt
%% and version 1.3 or later is part of all distributions of LaTeX
%% version 2003/12/01 or later.
%%
%% See CTAN archives in directory macros/latex/base/lppl.txt.
%%
%% CONTENTS:
%%   This work consists of the files dirtree.ins and dirtree.dtx.
%%   Derived files are dirtree.tex and dirtree.sty.
%%
%% DESCRIPTION:
%%   dirtree is a package displaying directory trees.
%%
% \fi
% \iffalse
%<latex-wrapper|tex> \def\fileversion{0.32}
%<latex-wrapper|tex> \def\filedate{2012/12/11}
%<*driver>
\documentclass{ltxdoc}
\GetFileInfo{dirtree.dtx}
\usepackage[T1]{fontenc}
\usepackage[latin1]{inputenc}
\usepackage{lmodern}
\usepackage[english]{babel}
\usepackage{hypdoc}
\usepackage{xcolor}
\usepackage{pstricks}
\usepackage{dirtree}
\newcommand*\package[1]{\textsf{#1}}
\newcommand*\file[1]{\texttt{#1}}
\AtBeginDocument{%
  \OnlyDescription % comment out for implementation details
  \EnableCrossrefs
  \RecordChanges
  \CodelineIndex}
\AtEndDocument{%
  \PrintChanges
  \PrintIndex}
\hbadness=7000     % Over and under full box warnings
\hfuzz=3pt
\begin{document}
  \DocInput{dirtree.dtx}
\end{document}
%</driver>
% \fi
% \changes{v0.01}{2004/08/10}{First realease to answer a question on
% fctt.}
% \changes{v0.11}{2005/12/19}{fix bug}
% \changes{v0.12}{2005/12/20}{\cs{DTbaselineskip}. local \cs{parskip},
%   \cs{baselineskip}, and \cs{strut} in order to fix a displaying bug.}
% \changes{v0.2}{2006/01/25}{dtx for CTAN, code for both Plain \TeX{}
%   and \LaTeX.}
% \changes{v0.3}{2009/04/08}{xkeyval syntax, breakable tree}
% \changes{v0.31}{2012/11/02}{bug about some lengths}
% \changes{v0.32}{2012/12/11}{bug about length (thanks to Philipp
% K\"uhl). Some macro names changed in order to prevent clash with
% other packages.}
%
% \DoNotIndex{\@,\@gobble,\@latex@error,\@nameuse}
% \DoNotIndex{\@ne,\@nil,\@namedef}
% \DoNotIndex{\advance}
% \DoNotIndex{\baselineskip,\box}
% \DoNotIndex{\catcode,\count@,\csname}
% \DoNotIndex{\def,\dimen,\dotfill,\dp}
% \DoNotIndex{\edef,\else,\endcsname,\endgraf,\expandafter}
% \DoNotIndex{\fi,\filedate,\fileversion}
% \DoNotIndex{\hbox,\hrule,\hsize,\ht}
% \DoNotIndex{\ifnum,\ifx,\input,\iterate}
% \DoNotIndex{\kern}
% \DoNotIndex{\leavevmode,\let,\long,\loop}
% \DoNotIndex{\m@ne,\message}
% \DoNotIndex{\NeedsTeXFormat,\newcount,\newcommand,\newdimen,\next,\noindent}
% \DoNotIndex{\par,\parindent,\parskip,\ProvidesFile,\ProvidesPackage}
% \DoNotIndex{\relax,\repeat,\rm,\rmfamily}
% \DoNotIndex{\setbox,\space,\strut}
% \DoNotIndex{\the,\tt,\ttfamily}
% \DoNotIndex{\vbox,\vrule,\vss}
% \DoNotIndex{\z@}
% \setcounter{IndexColumns}{2}
%
% \CheckSum{0}
% \CharacterTable
%  {Upper-case    \A\B\C\D\E\F\G\H\I\J\K\L\M\N\O\P\Q\R\S\T\U\V\W\X\Y\Z
%   Lower-case    \a\b\c\d\e\f\g\h\i\j\k\l\m\n\o\p\q\r\s\t\u\v\w\x\y\z
%   Digits        \0\1\2\3\4\5\6\7\8\9
%   Exclamation   \!     Double quote  \"     Hash (number) \#
%   Dollar        \$     Percent       \%     Ampersand     \&
%   Acute accent  \'     Left paren    \(     Right paren   \)
%   Asterisk      \*     Plus          \+     Comma         \,
%   Minus         \-     Point         \.     Solidus       \/
%   Colon         \:     Semicolon     \;     Less than     \<
%   Equals        \=     Greater than  \>     Question mark \?
%   Commercial at \@     Left bracket  \[     Backslash     \\
%   Right bracket \]     Circumflex    \^     Underscore    \_
%   Grave accent  \`     Left brace    \{     Vertical bar  \|
%   Right brace   \}     Tilde         \~}
% 
%  ................. Beginning of the documentation part .................
% 
% \title{The \package{dirtree} package\\Directory Tree}
% \author{Jean-C\^ome
% Charpentier\thanks{\texttt{Jean-Come.Charpentier@wanadoo.fr}}}
% \date{Version \fileversion\ \filedate\ \\
%       {\small Documentation revised \today}}
%
% \maketitle
%
% \begin{abstract}
%   Package \package{dirtree} allows to display directory tree, like
%   in the windows explorator.
% \end{abstract}
%
% \tableofcontents
% \section{Introduction}
% During a discussion on fctt (\texttt{fr.comp.text.tex}) about
% directory tree and how display such a structure, it appeared that
% there wasn't many packages which do the job.
%
% One obvious solution is to use \package{PSTricks} but some people
% don't like or don't know this package, so I made the first release
% of \package{dirtree}.
%
% In fact, I didn't plan to send it in CTAN but Robin Fairbairns and
% Danie was very convincing!
%
% \section{Usage}
% Package \package{dirtree} works both on Plain \TeX{} and \LaTeX{}.
% No surprise to call it:
% \begin{verbatim}
% \usepackage{dirtree}
% \end{verbatim}
% for \LaTeX{} and
% \begin{verbatim}
% \input dirtree
% \end{verbatim}
% for Plain \TeX{}. 
%
% Since version~0.3, \package{dirtree} has some
% package options. We'll see these options one by one.
%
% \begin{DescribeMacro}{\dirtree}
% The main macro is |\dirtree| which take one argument (the tree
% structure). This tree structure is a sequence of
% \begin{verbatim}
% .<level><space><text node>.<space>
% \end{verbatim}
% Note that there is a dot in the beginning and another one at the end
% of each node specification. The spaces are very important: if you
% forgot the space before the |level| there will be an error and if
% you forgot the space after the last dot, you don't indicate the end
% of the node. Since an end of line is like a space for \TeX{}, I
% recommand to write a node per line in the source file: it's handy
% and more readeable.
%
% The |level| indicates the node depth in the tree. There is two
% rules you must respect:
% \begin{enumerate}
% \item The root must have the level one.
% \item When you create a node, if the last node have the level $n$,
% the created node must have a level between 2 and $n+1$.
% \end{enumerate}
% In fact, you can indicates a level greater than $n+1$ if one node
% have a level $n$ somewhere in the tree but the result will be
% strange!
%
% A node of level $n$ will be connected to the last node defined which
% has a level lesser or equal to $n$.
%
% For example, the code
% \begin{verbatim}
% \dirtree{%
%  .1 /.
%  .2 bin.
%  .2 home.
%  .3 jeancome.
%  .4 texmf.
%  .5 tex.
%  .6 latex.
%  .7 dirtree.
%  .3 jeancomeson.
%  .3 jeancomedaughter.
%  .2 usr.
%  .3 bin.
%  .3 games.
%  .4 fortunes.
%  .3 include.
%  .3 local.
%  .4 bin.
%  .4 share.
%  .5 texmf.
%  .6 fonts.
%  .6 metapost.
%  .6 tex.
%  .3 share.
% }
% \end{verbatim}
% give the result
% \dirtree{%^^A
%  .1 /.
%  .2 bin.
%  .2 home.
%  .3 jeancome.
%  .4 texmf.
%  .5 tex.
%  .6 latex.
%  .7 dirtree.
%  .3 jeancomeson.
%  .3 jeancomedaughter.
%  .2 usr.
%  .3 bin.
%  .3 games.
%  .4 fortunes.
%  .3 include.
%  .3 local.
%  .4 bin.
%  .4 share.
%  .5 texmf.
%  .6 fonts.
%  .6 metapost.
%  .6 tex.
%  .3 share.
% }
%
% Note the |%| after the left brace in the beginning: it's important
% because the first character encountered must be a dot.
%
% \begin{DescribeMacro}{\DTstyle}
% A text node is typeset with the command |\DTstyle|. Its default
% value is |\ttfamily| when you are under \LaTeX{} and |\tt| when you
% are under Plain \TeX{}. You can redefine this macro as you want,
% it is used with the syntax |{\DTstyle{text node}}|, so you can use
% both |\ttfamily| and |\texttt| for example.
% \end{DescribeMacro}
%
% \begin{DescribeMacro}{\DTcomment}
% The |\DTcomment| command allows to put text at the right side, with
% leaders. The syntax is
% \begin{verbatim}
% \DTcomment{comment text}
% \end{verbatim}
% \begin{DescribeMacro}{\DTstylecomment}
% The style of comment is defined by |\DTstylecomment|. Its default
% value is |\rmfamilly| under \LaTeX{} and |\rm| under Plain \TeX{},
% and it acts like |\DTstyle|.
% Here is an example: the code
% \begin{verbatim}
% \renewcommand*\DTstylecomment{\rmfamily\color{green}\textsc}
% \renewcommand*\DTstyle{\ttfamily\textcolor{red}}
% \dirtree{%
%  .1 /.
%  .2 bin.
%  .2 home.
%  .3 jeancome.
%  .4 texmf.
%  .5 tex.
%  .3 jeancomeson\DTcomment{Guillaume}.
%  .3 jeancomedaughter\DTcomment{Mathilde}.
%  .2 usr.
%  .3 bin.
% }
% \end{verbatim}
% give the result
% \begingroup
% \renewcommand*\DTstylecomment{\rmfamily\color{green}\textsc}
% \renewcommand*\DTstyle{\ttfamily\textcolor{red}}
% \dirtree{%^^A
%  .1 /.
%  .2 bin.
%  .2 home.
%  .3 jeancome.
%  .4 texmf.
%  .5 tex.
%  .3 jeancomeson\DTcomment{Guillaume}.
%  .3 jeancomedaughter\DTcomment{Mathilde}.
%  .2 usr.
%  .3 bin.
% }
% \endgroup
% In this example we have used the \package{xcolor} package.
% \end{DescribeMacro}
% \end{DescribeMacro}
%
% You can build complex text node. For example, the code
% \begin{verbatim}
% \dirtree{%
%  .1 /.
%  .2 bin \ldots{} \begin{minipage}[t]{5cm}
%                    This directory holds executable files (binary
%                    files or link on binary files){.}
%         \end{minipage}.
%  .2 home \ldots{} \begin{minipage}[t]{5cm}
%                     jeancome\\
%                     guillaume\\
%                     mathilde\\
%                   \end{minipage}.
%  .4 texmf.
% }
% \end{verbatim}
% give the result
% \dirtree{%^^A
%  .1 /.
%  .2 bin \ldots{} \begin{minipage}[t]{5cm}
%                    This directory holds executable files (binary
%                    files or link on binary files){.}
%         \end{minipage}.
%  .2 home \ldots{} \begin{minipage}[t]{5cm}
%                     jeancome\\
%                     guillaume\\
%                     mathilde\\
%                   \end{minipage}.
%  .4 texmf.
% }
% We don't encourage to try too complicated code. Package
% \package{dirtree} is still fragile! Note that we pay attention to
% use optional parameter |[t]| in order to have a right vertical
% alignment with horizontal rules.
%
% \begin{DescribeMacro}{\DTsetlength}
% Some dimensions can be changed using the |\DTsetlength| command. The
% syntax is:
% \begin{verbatim}
% \DTsetlength{offset}{width}{sep}{rule-width}{dot-size}
% \end{verbatim}
% \begin{center}
%   \begin{pspicture}(0,0)(8,4)
%     \psset{linewidth=3pt}
%     \psline(1,0)(1,4)
%     \psline(1,2)(2.5,2)
%     \psline(3.5,0)(3.5,1.75)
%     \psset{linewidth=0.4pt}
%     \rput[bl](3,2){\framebox{\LARGE Text node}}
%     \psframe*(0.75,1.75)(1.25,2.25)
%     \psset{linewidth=0.2pt}
%     \psline{<->}(0.5,1.75)(0.5,2.25)
%     \uput[l](0.5,2){\scriptsize \ttfamily dot-size}
%     \psline{<->}(1,2.5)(2.5,2.5)
%     \uput[u](1.75,2.5){\scriptsize \ttfamily width}
%     \psline{<->}(2.5,2)(3,2)
%     \uput[u](2.75,2){\scriptsize \ttfamily sep}
%     \psline{<->}(3,1.5)(3.5,1.5)
%     \uput[d](3,1.5){\scriptsize \ttfamily offset}
%   \end{pspicture}
% \end{center}
% The default value are:
% \begin{itemize}
% \item \texttt{offset = 0.2em}
% \item \texttt{width = 1em}
% \item \texttt{sep = 0.2em}
% \item \texttt{rule-width = 0.4pt}
% \item \texttt{dot-size = 1.6pt}
% \end{itemize}
%
% \begin{DescribeMacro}{\DTbaselineskip}
% The last length parameter is |\DTbaselineskip| which indicates the
% skip between lines of the tree.
% \end{DescribeMacro}
%
% If we typeset the first example with
% \begin{verbatim}
% \setlength{\DTbaselineskip}{20pt}
% \DTsetlength{1em}{3em}{0.1em}{1pt}{4pt}
% \end{verbatim}
% we obtain the (strange) result:
% \begingroup
% \setlength{\DTbaselineskip}{20pt}
% \DTsetlength{1em}{3em}{0.1em}{1pt}{4pt}
% \dirtree{%^^A
%  .1 /.
%  .2 bin.
%  .2 home.
%  .3 jeancome.
%  .4 texmf.
%  .5 tex.
%  .6 latex.
%  .7 dirtree.
%  .3 jeancomeson.
%  .3 jeancomedaughter.
%  .2 usr.
%  .3 bin.
%  .3 games.
%  .4 fortunes.
%  .3 include.
%  .3 local.
%  .4 bin.
%  .4 share.
%  .5 texmf.
%  .6 fonts.
%  .6 metapost.
%  .6 tex.
%  .3 share.
% }
% \endgroup
% \end{DescribeMacro} %^^A of \DTsetlength
% \end{DescribeMacro} %^^A of \dirtree
%
% Note that \package{dirtree} package is not able to split tree on
% several pages. If this case occurs, the result will be very strange
% with overfull rules.  I suppose that the best is to place such trees
% inside floats.
%
% \section{ToDo}
%
% \begin{itemize}
% \item Parameters with \package{xkeyval} syntax;
% \item Command |\DTsplittree| to allows a tree to be typeseted on
% several pages;
% \item Style parameters to rules (color for example) and gap between
% text and comment (by now it's |\dotfill|).
% \item Dimension parameter |abovetreeskip| and |belowtreeskip|.
% \end{itemize}
%  .................... End of the documentation part ....................
%
%<*latex-wrapper>
% \newpage
% \section{\package{dirtree} \LaTeX{} Wrapper}
% Nothing special here but the |\DT@fromsty| definition. This latter
% is intended to check if \package{dirtree} is called under \LaTeX{}
% (with |\usepackage|) or under Plain \TeX{}.
%    \begin{macrocode}
\NeedsTeXFormat{LaTeX2e}[1995/06/01]
\ProvidesPackage{dirtree}[\filedate\space v\fileversion\space
                          package wrapper for dirtree]
\newcommand*\DT@fromsty{}
%%
%% This is file `dirtree.tex',
%% generated with the docstrip utility.
%%
%% The original source files were:
%%
%% dirtree.dtx  (with options: `tex')
%% 
%% IMPORTANT NOTICE:
%% 
%% For the copyright see the source file.
%% 
%% Any modified versions of this file must be renamed
%% with new filenames distinct from dirtree.tex.
%% 
%% For distribution of the original source see the terms
%% for copying and modification in the file dirtree.dtx.
%% 
%% This generated file may be distributed as long as the
%% original source files, as listed above, are part of the
%% same distribution. (The sources need not necessarily be
%% in the same archive or directory.)
%%
%% Package `dirtree.dtx'
%% -----------------------------------------------
%% Copyright (C) 2004-2009 Jean-C\^ome Charpentier
%% -----------------------------------------------
%%
%% This work may be distributed and/or modified under the
%% conditions of the LaTeX Project Public License, either version 1.3
%% of this license or (at your option) any later version.
%% The latest version of this license is in
%%   http://www.latex-project.org/lppl.txt
%% and version 1.3 or later is part of all distributions of LaTeX
%% version 2003/12/01 or later.
%%
%% See CTAN archives in directory macros/latex/base/lppl.txt.
%%
%% CONTENTS:
%%   This work consists of the files dirtree.ins and dirtree.dtx.
%%   Derived files are dirtree.tex and dirtree.sty.
%%
%% DESCRIPTION:
%%   dirtree is a package displaying directory trees.
%%
 \def\fileversion{0.32}
 \def\filedate{2012/12/11}
\message{`dirtree' v\fileversion, \filedate\space (jcc)}
\edef\DTAtCode{\the\catcode`\@}
\catcode`\@=11
\long\def\DT@loop#1\DT@repeat{%
  \def\DT@iterate{#1\relax\expandafter\DT@iterate\fi}%
  \DT@iterate
  \let\DT@iterate\relax
}
\let\DT@repeat=\fi
\expandafter\ifx\csname DT@fromsty\endcsname\relax
  \def\@namedef#1{\expandafter\def\csname #1\endcsname}
  \def\@nameuse#1{\csname #1\endcsname}
  \long\def\@gobble#1{}
\fi
\def\@nameedef#1{\expandafter\edef\csname #1\endcsname}
\newdimen\DT@offset \DT@offset=0.2em
\newdimen\DT@width \DT@width=1em
\newdimen\DT@sep \DT@sep=0.2em
\newdimen\DT@all
\DT@all=\DT@offset
\advance\DT@all \DT@width
\advance\DT@all \DT@sep
\newdimen\DT@rulewidth \DT@rulewidth=0.4pt
\newdimen\DT@dotwidth \DT@dotwidth=1.6pt
\newdimen\DTbaselineskip \DTbaselineskip=\baselineskip
\newcount\DT@counti
\newcount\DT@countii
\newcount\DT@countiii
\newcount\DT@countiv
\def\DTsetlength#1#2#3#4#5{%
  \DT@offset=#1\relax
  \DT@width=#2\relax
  \DT@sep=#3\relax
  \DT@all=\DT@offset
  \advance\DT@all by\DT@width
  \advance\DT@all by\DT@sep
  \DT@rulewidth=#4\relax
  \DT@dotwidth=#5\relax
}
\expandafter\ifx\csname DT@fromsty\endcsname\relax
  \def\DTstyle{\tt}
  \def\DTstylecomment{\rm}
\else
  \def\DTstyle{\ttfamily}
  \def\DTstylecomment{\rmfamily}
\fi
\def\DTcomment#1{%
  \kern\parindent\dotfill
  {\DTstylecomment{#1}}%
}
\newdimen\DT@indent
\newdimen\DT@parskip
\newdimen\DT@baselineskip
\def\dirtree#1{%
  \DT@indent=\parindent
  \parindent=\z@
  \DT@parskip=\parskip
  \parskip=\z@
  \DT@baselineskip=\baselineskip
  \baselineskip=\DTbaselineskip
  \let\DT@strut=\strut
  \def\strut{\vrule width\z@ height0.7\baselineskip depth0.3\baselineskip}%
  \DT@counti=\z@
  \let\next\DT@readarg
  \next#1\@nil
  \dimen\z@=\hsize
  \advance\dimen\z@ -\DT@offset
  \advance\dimen\z@ -\DT@width
  \setbox\z@=\hbox to\dimen\z@{%
    \hsize=\dimen\z@
    \vbox{\@nameuse{DT@body@1}}%
  }%
  \dimen\z@=\ht\z@
  \advance\dimen0 by\dp\z@
  \advance\dimen0 by-0.7\baselineskip
  \ht\z@=0.7\baselineskip
  \dp\z@=\dimen\z@
  \par\leavevmode
  \kern\DT@offset
  \kern\DT@width
  \box\z@
  \endgraf
  \DT@countii=\@ne
  \DT@countiii=\z@
  \dimen3=\dimen\z@
  \@namedef{DT@lastlevel@1}{-0.7\baselineskip}%
  \loop
  \ifnum\DT@countii<\DT@counti
    \advance\DT@countii \@ne
    \advance\DT@countiii \@ne
    \dimen\z@=\@nameuse{DT@level@\the\DT@countii}\DT@all
    \advance\dimen\z@ by\DT@offset
    \advance\dimen\z@ by-\DT@all
    \leavevmode
    \kern\dimen\z@
    \DT@countiv=\DT@countii
    \count@=\z@
    \DT@loop
      \advance\DT@countiv \m@ne
      \ifnum\@nameuse{DT@level@\the\DT@countiv} >
        \@nameuse{DT@level@\the\DT@countii}\relax
      \else
        \count@=\@ne
      \fi
    \ifnum\count@=\z@
    \DT@repeat
    \edef\DT@hsize{\the\hsize}%
    \count@=\@nameuse{DT@level@\the\DT@countii}\relax
    \dimen\z@=\count@\DT@all
    \advance\hsize by-\dimen\z@
    \setbox\z@=\vbox{\@nameuse{DT@body@\the\DT@countii}}%
    \hsize=\DT@hsize
    \dimen\z@=\ht\z@
    \advance\dimen\z@ by\dp\z@
    \advance\dimen\z@ by-0.7\baselineskip
    \ht\z@=0.7\baselineskip
    \dp\z@=\dimen\z@
    \@nameedef{DT@lastlevel@\the\DT@countii}{\the\dimen3}%
    \advance\dimen3 by\dimen\z@
    \advance\dimen3 by0.7\baselineskip
    \dimen\z@=\@nameuse{DT@lastlevel@\the\DT@countii}\relax
    \advance\dimen\z@ by-\@nameuse{DT@lastlevel@\the\DT@countiv}\relax
    \advance\dimen\z@ by0.3\baselineskip
    \ifnum\@nameuse{DT@level@\the\DT@countiv} <
        \@nameuse{DT@level@\the\DT@countii}\relax
      \advance\dimen\z@ by-0.5\baselineskip
    \fi
    \kern-0.5\DT@rulewidth
    \hbox{\vbox to\z@{\vss\hrule width\DT@rulewidth height\dimen\z@}}%
    \kern-0.5\DT@rulewidth
    \kern-0.5\DT@dotwidth
    \vrule width\DT@dotwidth height0.5\DT@dotwidth depth0.5\DT@dotwidth
    \kern-0.5\DT@dotwidth
    \vrule width\DT@width height0.5\DT@rulewidth depth0.5\DT@rulewidth
    \kern\DT@sep
    \box\z@
    \endgraf
  \repeat
  \parindent=\DT@indent
  \parskip=\DT@parskip
  \baselineskip=\DT@baselineskip
  \let\strut\DT@strut
}
\def\DT@readarg.#1 #2. #3\@nil{%
  \advance\DT@counti \@ne
  \@namedef{DT@level@\the\DT@counti}{#1}%
  \@namedef{DT@body@\the\DT@counti}{\strut{\DTstyle{#2}\strut}}%
  \ifx\relax#3\relax
    \let\next\@gobble
  \fi
  \next#3\@nil
}
\catcode`\@=\DTAtCode\relax
\endinput
%%
%% End of file `dirtree.tex'.

\ProvidesFile{dirtree.tex}
  [\filedate\space v\fileversion\space `dirtree' (jcc)]
%    \end{macrocode}
%</latex-wrapper>
%<*tex>
% \newpage
% \section{\package{dirtree} Code}
% An ``hello'' message.
%    \begin{macrocode}
\message{`dirtree' v\fileversion, \filedate\space (jcc)}
%    \end{macrocode}
% Save at current catcode and make @ a letter
%    \begin{macrocode}
\edef\DTAtCode{\the\catcode`\@}
\catcode`\@=11
%    \end{macrocode}
% \changes{v0.32}{2012/12/11}{LOOP, REPEAT and ITERATE modified to DT@
% form in order to prevent some clash with other packages.}
% Define |\DT@loop|, |\DT@repeat|, and |\DT@iterate| like |\loop|,
% |\repeat|, and |\iterate|. The |\DT@| form allows to place loop
% inside loop.
%    \begin{macrocode}
\long\def\DT@loop#1\DT@repeat{%
  \def\DT@iterate{#1\relax\expandafter\DT@iterate\fi}%
  \DT@iterate
  \let\DT@iterate\relax
}
\let\DT@repeat=\fi
%    \end{macrocode}
% Define some \LaTeX{} macros if we work under Plain \TeX{}.
% |\@namedef|-like for |\edef|.
%    \begin{macrocode}
\expandafter\ifx\csname DT@fromsty\endcsname\relax
  \def\@namedef#1{\expandafter\def\csname #1\endcsname}
  \def\@nameuse#1{\csname #1\endcsname}
  \long\def\@gobble#1{}
\fi
\def\@nameedef#1{\expandafter\edef\csname #1\endcsname}
%    \end{macrocode}
% Offset between vertical rule below text and text left boundary.
%    \begin{macrocode}
\newdimen\DT@offset \DT@offset=0.2em
%    \end{macrocode}
% Length of horizontal rule.
%    \begin{macrocode}
\newdimen\DT@width \DT@width=1em
%    \end{macrocode}
% Gap between horizontal rule and text.
%    \begin{macrocode}
\newdimen\DT@sep \DT@sep=0.2em
%    \end{macrocode}
% |\DT@offset|$\null+\null$|\DT@width|$\null+\null$|\DT@sep|
%    \begin{macrocode}
\newdimen\DT@all
\DT@all=\DT@offset
\advance\DT@all \DT@width
\advance\DT@all \DT@sep
%    \end{macrocode}
% Rule thickness
%    \begin{macrocode}
\newdimen\DT@rulewidth \DT@rulewidth=0.4pt
%    \end{macrocode}
% Size of square junction.
%    \begin{macrocode}
\newdimen\DT@dotwidth \DT@dotwidth=1.6pt
%    \end{macrocode}
% baselineskip inside tree.
%    \begin{macrocode}
\newdimen\DTbaselineskip \DTbaselineskip=\baselineskip
%    \end{macrocode}
% Max index node.
%    \begin{macrocode}
\newcount\DT@counti
%    \end{macrocode}
% Current index node
%    \begin{macrocode}
\newcount\DT@countii
%    \end{macrocode}
% |\DT@countiii|$\null=\null$|\DT@countii|$\null-1$. That is, Previous
% index node.
%    \begin{macrocode}
\newcount\DT@countiii
%    \end{macrocode}
% Last node of a level lesser or equal to current one.
%    \begin{macrocode}
\newcount\DT@countiv
%    \end{macrocode}
% \begin{macro}{\DTsetlength}
% |\DTsetlength| allows to define dimensions in use for the directory
% tree (see above).
%    \begin{macrocode}
\def\DTsetlength#1#2#3#4#5{%
  \DT@offset=#1\relax
  \DT@width=#2\relax
  \DT@sep=#3\relax
%    \end{macrocode}
% |\DT@all| is the width of a whole column.
%    \begin{macrocode}
  \DT@all=\DT@offset
  \advance\DT@all by\DT@width
  \advance\DT@all by\DT@sep
  \DT@rulewidth=#4\relax
  \DT@dotwidth=#5\relax
}
%    \end{macrocode}
% \end{macro}
% |\DTstyle| is the style used to typeset nodes. |\DTstylecomment| is
% the style used to typeset comments. Since \TeX{} and \LaTeX{} are
% very different, we test the format used before initializations.
% \begin{macro}{\DTstyle}
% \begin{macro}{\DTstylecomment}
%    \begin{macrocode}
\expandafter\ifx\csname DT@fromsty\endcsname\relax
  \def\DTstyle{\tt}
  \def\DTstylecomment{\rm}
\else
  \def\DTstyle{\ttfamily}
  \def\DTstylecomment{\rmfamily}
\fi
%    \end{macrocode}
% \end{macro}
% \end{macro}
% \begin{macro}{\DTcomment}
% |\DTcomment| places comment in a line of the tree.
%    \begin{macrocode}
\def\DTcomment#1{%
  \kern\parindent\dotfill
  {\DTstylecomment{#1}}%
}
%    \end{macrocode}
% \end{macro}
% \changes{v0.31}{2012/11/02}{save lengths as lengths (not as macro)}
% In order to save some lengths we create newdimen
%    \begin{macrocode}
\newdimen\DT@indent
\newdimen\DT@parskip
\newdimen\DT@baselineskip
%    \end{macrocode}
% \begin{macro}{\dirtree}
% |\dirtree| is the main package macro.
%    \begin{macrocode}
\def\dirtree#1{%
%    \end{macrocode}
% Change some parameters (save them before).
%    \begin{macrocode}
  \DT@indent=\parindent
  \parindent=\z@
  \DT@parskip=\parskip
  \parskip=\z@
  \DT@baselineskip=\baselineskip
  \baselineskip=\DTbaselineskip
  \let\DT@strut=\strut
  \def\strut{\vrule width\z@ height0.7\baselineskip depth0.3\baselineskip}%
%    \end{macrocode}
% Read the argument and before that, initialize counters. |\DTcounti|
% is the current index node.
%    \begin{macrocode}
  \DT@counti=\z@
  \let\next\DT@readarg
  \next#1\@nil
%    \end{macrocode}
% When |\DT@readarg| has done its job, the node levels and the node
% texts are saved in |\DT@level@<index>| and |\DT@body@<index>|
% respectively. |\DT@counti| holds the greater index. We can now
% display the tree.
%
% Firstly, display the root. For that, the text is boxed.
%    \begin{macrocode}
  \dimen\z@=\hsize
  \advance\dimen\z@ -\DT@offset
  \advance\dimen\z@ -\DT@width
  \setbox\z@=\hbox to\dimen\z@{%
    \hsize=\dimen\z@
    \vbox{\@nameuse{DT@body@1}}%
  }%
%    \end{macrocode}
% We change the height and the depth of this box in order to have the
% same total height and a height of |0.7\baselineskip|, that is, the
% height of |\strut|.
%    \begin{macrocode}
  \dimen\z@=\ht\z@
  \advance\dimen0 by\dp\z@
  \advance\dimen0 by-0.7\baselineskip
  \ht\z@=0.7\baselineskip
  \dp\z@=\dimen\z@
%    \end{macrocode}
% Then we display this box with an indentation as if there had a level 0.
%    \begin{macrocode}
  \par\leavevmode
  \kern\DT@offset
  \kern\DT@width
  \box\z@
  \endgraf
%    \end{macrocode}
% Initialize index for the loop.
%    \begin{macrocode}
  \DT@countii=\@ne
  \DT@countiii=\z@
%    \end{macrocode}
% |\dimen3| holds the height of the node in the tree. In fact, the
% bottom of the node since this dimension is used to connect vertical
% rules.
%    \begin{macrocode}
  \dimen3=\dimen\z@
%    \end{macrocode}
% |\DT@lastlevel@<level>| holds the baseline of the last node in level
% |<level>|.
%    \begin{macrocode}
  \@namedef{DT@lastlevel@1}{-0.7\baselineskip}%
%    \end{macrocode}
% Loop for displaying the remainder of the tree.
%    \begin{macrocode}
  \loop
%    \end{macrocode}
% Exit loop when the last current index is lesser or equal to max
% index.
%    \begin{macrocode}
  \ifnum\DT@countii<\DT@counti
%    \end{macrocode}
% |\DT@counti| holds current index and |\DT@countii| holds previous
% index (just current index minus one).
%    \begin{macrocode}
    \advance\DT@countii \@ne
    \advance\DT@countiii \@ne
%    \end{macrocode}
% Horizontal offset for the text:
%
% $(\mathtt{current\ level} - 1) \times \mathtt{DT@all} +
% \mathtt{DT@offset}$.
%    \begin{macrocode}
    \dimen\z@=\@nameuse{DT@level@\the\DT@countii}\DT@all
    \advance\dimen\z@ by\DT@offset
    \advance\dimen\z@ by-\DT@all
    \leavevmode
    \kern\dimen\z@
%    \end{macrocode}
% Look for last node in previous level in order to know how connect
% the current node.
%    \begin{macrocode}
    \DT@countiv=\DT@countii
    \count@=\z@
    \DT@loop
%    \end{macrocode}
% Look for previous node
%    \begin{macrocode}
      \advance\DT@countiv \m@ne
%    \end{macrocode}
% Repeat until this previous node has a level lesser or equal to
% current level.
%    \begin{macrocode}
      \ifnum\@nameuse{DT@level@\the\DT@countiv} >
        \@nameuse{DT@level@\the\DT@countii}\relax
      \else
        \count@=\@ne
      \fi
    \ifnum\count@=\z@
    \DT@repeat
%    \end{macrocode}
% Now |\DT@countiv| holds the index node connected to current node.
%
% We box the text node.
%    \begin{macrocode}
    \edef\DT@hsize{\the\hsize}%
%    \end{macrocode}
%    \begin{macrocode}
    \count@=\@nameuse{DT@level@\the\DT@countii}\relax
%    \end{macrocode}
% Since text node is vboxed, we use a |\hsize| minus horizontal
% current offset.
%    \begin{macrocode}
    \dimen\z@=\count@\DT@all
    \advance\hsize by-\dimen\z@
    \setbox\z@=\vbox{\@nameuse{DT@body@\the\DT@countii}}%
%    \end{macrocode}
% Restore |\hsize|.
%    \begin{macrocode}
    \hsize=\DT@hsize
%    \end{macrocode}
% Change height and depth in such a way that height is
% |0.7\DT@baselineskip| (that is, the |\strut| height), and total
% height is unchanged.
%    \begin{macrocode}
    \dimen\z@=\ht\z@
    \advance\dimen\z@ by\dp\z@
    \advance\dimen\z@ by-0.7\baselineskip
    \ht\z@=0.7\baselineskip
    \dp\z@=\dimen\z@
%    \end{macrocode}
% Save the height of the box in tree. The last node is the last node
% in its level!
%    \begin{macrocode}
    \@nameedef{DT@lastlevel@\the\DT@countii}{\the\dimen3}%
%    \end{macrocode}
% |\dimen3| holds the vertical position of the bottom.
%    \begin{macrocode}
    \advance\dimen3 by\dimen\z@
    \advance\dimen3 by0.7\baselineskip
%    \end{macrocode}
% Display vertical rule
%    \begin{macrocode}
    \dimen\z@=\@nameuse{DT@lastlevel@\the\DT@countii}\relax
    \advance\dimen\z@ by-\@nameuse{DT@lastlevel@\the\DT@countiv}\relax
    \advance\dimen\z@ by0.3\baselineskip
%    \end{macrocode}
% If this vertical rule connect two nodes which have different level,
% the rule must be reduced by |0.5\baselineskip| (the rule don't rise
% up to the baselineskip of the connected node).
%    \begin{macrocode}
    \ifnum\@nameuse{DT@level@\the\DT@countiv} <
        \@nameuse{DT@level@\the\DT@countii}\relax
      \advance\dimen\z@ by-0.5\baselineskip
    \fi
%    \end{macrocode}
% Display vertical rule
%    \begin{macrocode}
    \kern-0.5\DT@rulewidth
    \hbox{\vbox to\z@{\vss\hrule width\DT@rulewidth height\dimen\z@}}%
    \kern-0.5\DT@rulewidth
%    \end{macrocode}
% Display square junction.
%    \begin{macrocode}
    \kern-0.5\DT@dotwidth
    \vrule width\DT@dotwidth height0.5\DT@dotwidth depth0.5\DT@dotwidth
    \kern-0.5\DT@dotwidth
%    \end{macrocode}
% Display horizontal rule and gap between horizontal rule and text node.
%    \begin{macrocode}
    \vrule width\DT@width height0.5\DT@rulewidth depth0.5\DT@rulewidth
    \kern\DT@sep
%    \end{macrocode}
% Display text node.
%    \begin{macrocode}
    \box\z@
%    \end{macrocode}
% New paragraph for next node.
%    \begin{macrocode}
    \endgraf
  \repeat
%    \end{macrocode}
% Restore indentation,  paragraph skip, and |\strut|.
%    \begin{macrocode}
  \parindent=\DT@indent
  \parskip=\DT@parskip
%    \end{macrocode}
% \changes{v0.32}{2012/12/11}{Inverse order of assignation between
% baselineskip and DT@baselineskip.}
%    \begin{macrocode}
  \baselineskip=\DT@baselineskip
  \let\strut\DT@strut
}
%    \end{macrocode}
% \end{macro}
% \begin{macro}{\DT@readarg}
% The first processing step is to read the whole tree. It's a
% recursive macro: each call process one node, that is, a
% \begin{verbatim}
% .<index> <text node>.<space>
% \end{verbatim}
% in the source file.
%    \begin{macrocode}
\def\DT@readarg.#1 #2. #3\@nil{%
%    \end{macrocode}
% |\DT@counti| is the current index.
%    \begin{macrocode}
  \advance\DT@counti \@ne
%    \end{macrocode}
% save level node in |\DT@level@<index>| and text node in
% |\DT@body@<index>|. Two \package{dirtree} |\strut| are added to text
% node in order to ensure a right vertical alignment, according to
% \package{dirtree} |\baselineskip|
%    \begin{macrocode}
  \@namedef{DT@level@\the\DT@counti}{#1}%
  \@namedef{DT@body@\the\DT@counti}{\strut{\DTstyle{#2}\strut}}%
%    \end{macrocode}
% If |#3| is not empty, it contains the remainder nodes and macro
% calls itself. Otherwise, recursive call is stopped.
%    \begin{macrocode}
  \ifx\relax#3\relax
    \let\next\@gobble
  \fi
  \next#3\@nil
}
%    \end{macrocode}
% \end{macro}
% Restore at catcode.
%    \begin{macrocode}
\catcode`\@=\DTAtCode\relax
%    \end{macrocode}
%</tex>
% \endinput
