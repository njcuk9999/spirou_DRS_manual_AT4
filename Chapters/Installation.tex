% ---------------------------------------------------------------------
\chapter{Installation}
% ---------------------------------------------------------------------

This is just for use with the current installation files (version 43), the author does not recommend that this is the final set-up procedure, nor that any of the `modifications' to the original source code applied here should be used in any final version of the DRS (better solutions should be found). \\


% ---------------------------------------------------------
\section{Activate environment}
\label{section:activate}
% ---------------------------------------------------------

If you followed the steps in \ref{section:using_setup_sh} please follow this section, if however you followed the steps in \ref{section:using_bashrc} continue to Section \ref{section:downloading_install_Scripts}.

\noindent Before running the installation (or before running the code) one must run the following:
\begin{lstlisting}[style=bashstyle]
@source@ env_setup.sh
\end{lstlisting}

\noindent Output should look like this:
\begin{lstlisting}[style=bashstyle]
  =========================================
    Environmental setup for SPIRou Pipeline
  =========================================
    Setting up environment
    Set up for {INSTRUMENT}
      - Python located at: {PYTHON_DIR}
      - GLS located at: {GSL_DIR}
      - data located at:
            {DATA_ROOT}
            {DATA_RAW_ROOT}
            {DATA_ROOT_REDUCED}
            {DATA_ROOT_CALIB}
            {DATA_ROOT_MSG}
            {DATA_ROOT_TMP}

    Done
\end{lstlisting}

\noindent Note to deactivate type
\begin{lstlisting}[style=bashstyle]
@source@ env_setup.sh --clean
\end{lstlisting}


% ---------------------------------------------------------
\section{Downloading the preparing the install scripts}
\label{section:downloading_install_Scripts}
% ---------------------------------------------------------

Minor modifications need to be made to the code to allow a isolated version of GSL to be used. If and only if your GSL is installed to `/opt/gsl/' will the code install correctly with out this modification. The other steps are as in the original installation procedure. \\

\begin{enumerate}
\item change directory to \definevariable{DIR}
\begin{lstlisting}[style=bashstyle]
{DIR}
\end{lstlisting}

\item download source code for spirou, location should be \definevariable{DIR}/spirou
\begin{lstlisting}[style=bashstyle]
svn co https://svn.lam.fr/repos/spirou/trunk spirou
\end{lstlisting}

\end{enumerate}

\begin{enumerate}
\item change directory to source files
\begin{lstlisting}[style=bashstyle]
cd {DIR}/spirou/src
\end{lstlisting}

\item find and change \definevariable{DIR}/spirou/src/C/setup.py

\begin{enumerate}
\item Below the line \definevariable{PYTHON\_INCLUDE\_DIR} = os.getenv(`\definevariable{PYTHON\_INCLUDE\_DIR}') add the following:
\begin{lstlisting}[style=text]
<gsl_include_dir> = os.getenv('GSL_INCLUDE_DIR')
<gsl_library_dir> = os.getenv('GSL_LIBRARY_DIR')
\end{lstlisting}

\item Find and replace all instances of
\begin{lstlisting}[style=text]
include_dirs = [python_include_dir,@'/opt/gsl/include'@],
\end{lstlisting}

with
\begin{lstlisting}[style=text]
include_dirs = [python_include_dir, @gsl_include_dir@],
\end{lstlisting}

\item Find and replace all instances of 
\begin{lstlisting}[style=text]
library_dirs = [@'/opt/gsl/lib'@],  
\end{lstlisting}

with
\begin{lstlisting}[style=text]
library_dirs = [@gsl_library_dir@],
\end{lstlisting}

\end{enumerate}

\end{enumerate}

% ---------------------------------------------------------
\section{Running installation scripts}
\label{section:running_install_Scripts}
% ---------------------------------------------------------

run installation scripts

\begin{lstlisting}[style=bashstyle]
./hardrsInstall SPIROU 2.7
./scriptInstall SPIROU 2.7
\end{lstlisting}

% ---------------------------------------------------------
\section{Subversion}
\label{section:how_to_update}
% ---------------------------------------------------------

\subsection{Updating local version}
The code currently uses `Subversion' (SVN) to monitor the different versions of the code that is under development. To update a local version of the DRS:

\begin{enumerate}
\item Go inside the \definevariable{DIR}/spirou folder:
\begin{lstlisting}[style=bashstyle]
cd \{DIR\}/spirou
\end{lstlisting}

\item Type:
\begin{lstlisting}[style=bashstyle]
svn update
\end{lstlisting}

You will then receive information on the DRS version.

\item Then you need to re-install the DRS. \textcolor{red}{This will rewrite all codes in the \{INSTALL\_ROOT\} folder}.

Reinstall the DRS using:
\begin{lstlisting}[style=bashstyle]
cd src
./hardrsInstall SPIROU 2.7
./scriptInstall SPIROU 2.7
\end{lstlisting}

\end{enumerate}

You are now ready to work with the latest version.


% ---------------------------------------------------------
\subsection{Adding new files to the shared DRS}
\label{section:how_to_add}
% ---------------------------------------------------------

Note: this is only to be done if your new functions are running correctly.

\begin{enumerate}
\item Go inside the \definevariable{DIR}/spirou folder:
\begin{lstlisting}[style=bashstyle]
cd \{DIR\}/spirou
\end{lstlisting}

\item run the update on the svn
\begin{lstlisting}[style=bashstyle]
svn update
\end{lstlisting}

\item copy your modifications into the source code:
\begin{lstlisting}[style=bashstyle]
cp {modified file} src/{location of new file}
\end{lstlisting}

\item Add the file to the list of pending SVN files:
\begin{lstlisting}[style=bashstyle]
svn add src/{location of new file}
\end{lstlisting}

\item once all new files are added, commit to releasing the modifications to the SVN:
\begin{lstlisting}[style=bashstyle]
svn commit -m 'modification of {file name}'
\end{lstlisting}

\end{enumerate}

\subsection{Useful links about subversion}
\label{section:useful-links}

\begin{itemize}
\item resentation: \url{http://multithread.org/files/presentations/subversion/subversion_tutorial.pdf}
\item Book: \url{http://svnbook.red-bean.com/}
\item Introduction to subversion: \url{https://dev.nozav.org/intro_svn.html}
\end{itemize}