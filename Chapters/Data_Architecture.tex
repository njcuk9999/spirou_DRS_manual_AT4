\chapter{Data Architecture}

Below we describe the data architecture, first by providing the installed file structure in a tree diagram (\ref{section:folder_layout}), and then describing individual directories.

% ---------------------------------------------------------
\section{Installed file structure}
\label{section:folder_layout}
% ---------------------------------------------------------

The file structure should look as follows:
\customdirtree{%
.1 \{DIR\}.
.2 \{INSTALL\_ROOT\}.
.3 bin.
.4 \DTcomment{Recipe symbolic-link}.
.3 DRS\_\{INSTRUMENT\_NAME\}.
.4 \DTcomment{Worker functions}.
.2 spirou.
.3 read\_harps.py.
.3 src.
.4 \DTcomment{source installation files}.
.1 \{DATA\_ROOT\}*.
.2 raw.
.3 YYYYMMDD \DTcomment{Observation directory}.
.4 \DTcomment{Raw observation files}. 
.2 reduced.
.2 calibDB.
.2 msg.
.2 tmp.
}

* Note that this is the recommended file structure and raw, reduced, calibDB, msg and tmp can be changed using the \definevariable{DATA\_ROOT\_RAW}, \definevariable{DATA\_ROOT\_REDUCED}, \definevariable{DATA\_ROOT\_CALIB}, 
\definevariable{DATA\_ROOT\_MSG}, and \definevariable{DATA\_ROOT\_TMP} variables in \ref{section:edit_env_setup}.

\newpage

\noindent i.e. for the given `env\_setup.sh' (Appendix \ref{section:source_code_env_setup_sh}) this would be:
\customdirtree{%
.1 /data/spirou/drs.
.2 INTROOT/.
.3 bin.
.4 \DTcomment{Recipe symbolic-link}.
.3 DRS\_SPIROU.
.4 \DTcomment{Worker functions}.
.2 data.
.3 raw.
.4 YYYYMMDD \DTcomment{night\_repository}.
.5 \DTcomment{Raw observation files}. 
.3 reduced.
.3 calibDB.
.3 msg.
.3 tmp.
.2 spirou.
.3 read\_harps.py.
.3 src.
.4 \DTcomment{source installation files}.
}

% ------------------------------------------------------------------------
\section{The Installation root directory}
\label{section:install_root_folder}
% ------------------------------------------------------------------------

The \definevariable{INSTALL\_ROOT} contains all the installed recipes, worker functions and configuration files needed to run the DRS. The file structure is set up as below: 

\customdirtree{%
.1 /data/spirou/drs.
.2 \{INSTALL\_ROOT\}.
.3 bin.
.4 \DTcomment{Recipes}.
.3 DRS\_SPIROU.
}

% ------------------------------------------------------------------------
\subsection{The bin directory}
\label{section:install_root_folder:bin_folder}
% ------------------------------------------------------------------------

The bin directory is located in the \definevariable{INSTALL\_ROOT} directory. This contains symbolic links to the DRS recipes.

\newpage

% ------------------------------------------------------------------------
\subsection{The DRS\_SPIROU directory}
\label{section:install_root_folder:drs_spirou}
% ------------------------------------------------------------------------

The DRS\_SPIROU directory contains all the worker functions and configuration files for the DRS. The file structure is as follows:

\customdirtree{%
.1 DRS\_SPIROU.
.2 C.
.3 \DTcomment{The C programs and setup.py}.
.2 config.
.3 \DTcomment{The config files}.
.2 docs.
.2 fortran.
.3 \DTcomment{The fortran programs and install.csh}.
.2 man.
.2 python.
.3 C-modules.
.3 f2pymodules.
.3 Modules.
.3 Recipes.
.2 util.
}

The C and Fortran functions are found in their own directories. The configuration files are explained in Section \ref{section:the_config_files}.

The python directory contains the translation of the C and fortran functions into python (C-modules and f2pymodules directories).

The functions `startup.py' and `startup\_recipes.py' are called automatically by the recipes.

`python.csh' is the custom python environment for the DRS.

The python modules are explained in Section \ref{section:the_modules} and the recipes are explained in Section \ref{section:the_recipes}.

% ------------------------------------------------------------------------
\section{The data root directory}
\label{section:data_root_folder}
% ------------------------------------------------------------------------

This is the directory where all the data should be stored. The default and recommended design is to have \definevariable{DATA\_ROOT\_RAW}, \definevariable{DATA\_ROOT\_REDUCED}, \definevariable{DATA\_ROOT\_CALIB}, 
\definevariable{DATA\_ROOT\_MSG}, and \definevariable{DATA\_ROOT\_TMP} as sub-directories of \definevariable{DATA\_ROOT}. However as in Section \ref{section:edit_env_setup} these sub-directories can be defined elsewhere.

\newpage

% ------------------------------------------------------------------------
\subsection{The raw and reduced data directories}
\label{section:data_root_folder:raw_folder}
% ------------------------------------------------------------------------
The raw observed data is stored under the \definevariable{DATA\_ROOT\_RAW} path, the files are stored by night in the form \constantFolderDateFormat. \\

\noindent The file structure can be seen below:
\customdirtree{%
.1 \{DATA\_ROOT\_RAW\}.
.2 YYYYMMDD \DTcomment{night\_repository}.
.3 \DTcomment{Raw observation files}. 
.3 dark\_dark\{name\}.fits.
.3 dark\_flat\{name\}.fits.
.3 flat\_dark\{name\}.fits.
.3 fp\_fp\{name\}.fits.
}

% ------------------------------------------------------------------------
\section{The calibration database directory}
\label{section:data_root_folder:calibDB}
% ------------------------------------------------------------------------

\customdirtree{%
.1 \{DATA\_ROOT\}.
.2 calibDB or \{DATA\_ROOT\_CALIB\}.
.3 master\_calib\_SPIROU.txt.
.3 \DTcomment{The calibration fits files}. 
}

The calibDB contains all the calibration files that pass the qulaity tests and a test file `master\_calib.txt'. It is located at \definevariable{DATA\_ROOT\_CALIB} or if this is not defined is located by default at the \definevariable{DATA\_ROOT} directory.

Each line in this file is a uniqie calibration file and lines are formatted in the following manner:
\begin{lstlisting}[style=text]
@{key}@ @{night_repository}@ @{filename}@ @{human readable date}@ @{unix time}@ 
\end{lstlisting}

\noindent where
\begin{itemize}
\item \definekeyword{key} is a code assigned for each type of calibration file. Currently accepted keys are:
\begin{itemize}
\item DARK - Created from cal\_DARK\_spirou
\item ORDER\_PROFIL\_\definevariable{fiber} - Created in cal\_loc\_RAW\_spirou
\item LOC\_C - Created in cal\_loc\_RAW\_spirou
\item TILT - Created in cal\_SLIT\_spirou
\item FLAT\_\definevariable{fiber} - Created in cal\_FF\_RAW\_spirou
\item WAVE - Currently manually added
\end{itemize}

\item \definekeyword{night\_repository} is the raw data observation directory (in \definevariable{DATA\_ROOT\_RAW}) normally in the form \constantFolderDateFormat.

\item \definekeyword{filename} is the filename of the calibration file (located in the calibDB).

\item \definekeyword{human readable date} is the date in DD/MM/YY/HH:MM:SS format taken from the header of the file that created the calibration file (using the header keyword `\constantAcqtimeKey').

\item \definekeyword{unix time} is the time (as in \definekeyword{human readable date}) but in unix time (in seconds).

\end{itemize}

\noindent An example working master\_calib\_SPIROU.txt is shown below (assuming the listed files are present in \definevariable{DATA\_ROOT\_CALIB})
\begin{lstlisting}[style=text]
DARK 20170710 dark_dark02d406.fits 07/10/17/16:37:48 1499704668.0
ORDER_PROFIL_C 20170710 dark_flat02f10_order_profil_C.fits 07/10/17/17:03:50 1499706230.0
LOC_C 20170710 dark_flat02f10_loco_C.fits 07/10/17/17:03:50 1499706230.0
ORDER_PROFIL_AB 20170710 flat_dark02f10_order_profil_AB.fits 07/10/17/17:07:08 1499706428.0
LOC_AB 20170710 flat_dark02f10_loco_AB.fits 07/10/17/17:07:08 1499706428.0
TILT 20170710 fp_fp02a203_tilt.fits 07/10/17/17:25:15 1499705515.0
FLAT_C 20170710 dark_flat02f10_flat_C.fits 07/10/17/17:03:50 1499706230.0
WAVE 20170710 spirou_wave_ini3.fits 07/10/17/17:03:50 1499706230.0
\end{lstlisting}

