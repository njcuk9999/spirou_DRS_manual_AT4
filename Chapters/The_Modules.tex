% ---------------------------------------------------------------
\chapter{The Modules}
\label{section:the_modules}
% ---------------------------------------------------------------

Some text here

% ---------------------------------------------------------------
\section{The Module directory}
% ---------------------------------------------------------------

This contains all the python worker files for the DRS. The file structure is as below.

\customdirtree{%
.1 \{DIR\}.
.2 \{INSTALL\_ROOT\}.
.3 DRS\_SPIROU.
.4 python.
.5 Modules.
.6 hadgtVISU.
.7 \DTcomment{hadmrVISU module files}.
.6 hadmrBACK.
.7 \DTcomment{hadmrBACK module files}.
.6 hadmrBIAS.
.7 \DTcomment{hadmrBIAS module files}.
.6 hadmrCDB.
.7 \DTcomment{hadmrCDB module files}.
.6 hadmrDARK.
.7 \DTcomment{hadmrDARK module files}.
.6 hadmrEXTOR.
.7 \DTcomment{hadmrEXTOR module files}.
.6 hadmrFITS.
.7 \DTcomment{hadmrFITS module files}.
.6 hadmrFLAT.
.7 \DTcomment{hadmrFLAT module files}.
.6 hadmrLED.
.7 \DTcomment{hadmrLED module files}.
.6 hadmrLOCOR.
.7 \DTcomment{hadmrLOCOR module files}.
.6 hadmrMATH.
.7 \DTcomment{hadmrMATH module files}.
.6 hadmrRV.
.7 \DTcomment{hadmrRV module files}.
.6 hadmrSIMU.
.7 \DTcomment{hadmrSIMU module files}.
.6 hadmrTHORCA.
.7 \DTcomment{hadmrTHORCA module files}.
.6 hadrgdCONFIG.
.7 \DTcomment{hadrgdCONFIG module files}.
.6 spirouBACK.py.
.6 spirouBIAS.py.
.6 spirouCDB.py.
.6 spirouEXTOR.py.
.6 spirouFITS.py.
.6 spirouFLAT.py.
.6 spirouLOCOR.py.
.6 spirouRV.py.
.6 spirouTHORCA.py.
.6 spirouVISU.py.
}

% -----------------------------------------------------------------------------------
\clearpage
\newpage
\section{The spirouBACK module}
% -----------------------------------------------------------------------------------

Contains functions to calculate the detector background

\subsection{measure\_bkgr}

\begin{lstlisting}[style=pythonstyle]
@measure_bkgr@(data, order_profile, size, ccdgain, ccdsigdet)
    """
    Measures the background via interpolation over many small boxes of
         width/height="size" uses the order_profile to mask out (?) the orders
         TODO: Is that what this is doing?

    :param data: 2D numpy array, the image to measure the background of
    :param order_profile: 2D numpy array, the smoothed image using the order
                          fits
    :param size: int, size of the sub-frame to measure the background in
    :param ccdgain: float, the gain of the image
    :param ccdsigdet: float, the sigdet of the image

    :return bkgr: 2D numpy array, the interpolated background image
    :return xc: numpy array (size x data.shape[0]) in steps of 2*size
    :return yc: numpy array (size x data.shape[1]) in steps of 2*size
    :return mode: numpy array (size x size) the mode (?) of each box
    """
\end{lstlisting}

\noindent Note: Not 100\% sure how this function works. \\

\subsection{measure\_bkgr2}

\begin{lstlisting}[style=pythonstyle]
@measure_bkgr2@(data, order_profile, size, ccdgain, ccdsigdet)
    """
    Measures the background via interpolation over many small boxes of
         width/height="size" uses the order_profile to mask out (?) the orders
         TODO: Is that what this is doing?

    :param data: 2D numpy array, the image to measure the background of
    :param order_profile: 2D numpy array, the smoothed image using the order
                          fits
    :param size: int, size of the sub-frame to measure the background in
    :param ccdgain: float, the gain of the image
    :param ccdsigdet: float, the sigdet of the image

    :return bkgr: 2D numpy array, the interpolated background image
    :return xc: numpy array (size x data.shape[0]) in steps of 2*size
    :return yc: numpy array (size x data.shape[1]) in steps of 2*size
    :return mode: numpy array (size x size) the mode (?) of each box
    :return binc: numpy array, the bins the histogram is binned along
    :return histo: numpy array, the histogram of each subframe
    """
\end{lstlisting}

\noindent Note: Not 100\% sure how this function works. \\

\subsection{measure\_bkgr\_FF}

\begin{lstlisting}[style=pythonstyle]
@measure_bkgr_FF@(data, size, ccdgain, ccdsigdet)
    """
    Measures the background via interpolation over many small boxes of 
         width/height="size"
    
    :param data: 2D numpy array, the image to measure the background of
    :param size: int, size of the sub-frame to measure the background in
    :param ccdgain: float, the gain of the image
    :param ccdsigdet: float, the sigdet of the image
    
    :return bkgr: 2D numpy array, the interpolated background image
    :return xc: numpy array (size x data.shape[0]) in steps of 2*size
    :return yc: numpy array (size x data.shape[1]) in steps of 2*size
    :return minlevel: numpy array (size x size) - the central value of each
                      subframed box
    """
\end{lstlisting}


\subsection{measure\_bkgr\_FF2}

\begin{lstlisting}[style=pythonstyle]
@measure_bkgr_FF2@(data, size, ccdgain, ccdsigdet, seuil)
    """
    Measures the background via interpolation over many small boxes of 
         width/height="size" uses a threshold="seuil" above which the value is
         set to the max value in a box

    :param data: 2D numpy array, the image to measure the background of
    :param size: int, size of the sub-frame to measure the background in
    :param ccdgain: float, the gain of the image
    :param ccdsigdet: float, the sigdet of the image
    "param seuil: float, a threshold for 

    :return bkgr: 2D numpy array, the interpolated background image
    :return xc: numpy array (size x data.shape[0]) in steps of 2*size
    :return yc: numpy array (size x data.shape[1]) in steps of 2*size
    :return minlevel: numpy array (size x size) - the central value of each
                      subframed box
    """
\end{lstlisting}


% -----------------------------------------------------------------------------------
\clearpage
\newpage
\section{The spirouCDB module}
% -----------------------------------------------------------------------------------

Contains functions for the calibrations database (infos for master\_calib.txt, and to coy files in the caliDB directory)

\subsection{filename2tunix}
\begin{lstlisting}[style=pythonstyle]
@filename2tunix@(cfht_name)
    """
    Open the fits file "cfht_name" read the header key 'DATEEND' and return
    it as a unix timestamp
    
    :param cfht_name: string, the fits filename and location of file
    :return tt: float, the unix time of "cfht_name" store in 'DATEEND' 
    """
\end{lstlisting}

\noindent Note: Currently not used in code \\

\subsection{filename2tunix2}
\begin{lstlisting}[style=pythonstyle]
@filename2tunix2@(cfht_name)
    """
    Open the fits file "cfht_name" read the header key 'ACQTIME1' and return
    it as a unix timestamp

    :param cfht_name: string, the fits filename and location of file
    :return tt: float, the unix time of "cfht_name" store in 'ACQTIME1'
    """
\end{lstlisting}

\subsection{filename2tuni\_old}
\begin{lstlisting}[style=pythonstyle]
@filename2tunix_old@(file_name)
    """
    Extracts the timestamp information from a filename and converts it to a 
        unix timestamp

    :param file_name: string, filename toe look for timestamp in, must contain
                      *.YY-MM-DDThh:mm:ss_*  or *.YY-MM-DDThh:mm:ss.fits
                      i.e. filename.17-06-12T15:32:42
                      
    :return tt: float, the unix time from file_name
    """
\end{lstlisting}

\noindent Note: Currently not used in code \\

\subsection{data2tunix}
\begin{lstlisting}[style=pythonstyle]
@date2tunix@(argdate)
    """
    Converts a string date in format YY-MM-DD into a unix timestamp

    :param argdate: string in form YY-MM-DD
    
    :return tt: float, the unix time from argdate
    """
\end{lstlisting}

\noindent Note: Is this used? \\

\subsection{update\_master}
\begin{lstlisting}[style=pythonstyle]
@update_master@(master_file, night, key, file_name, opt_cmt)
    """
    Update calibration database with new row in form:

    {key} {night respoitory} {filename} {human readable time} {unix time}
    
    Note calibration database is locked while this function is active
    Note time comes from file_name so must have header 'ACQTIME1'

    :param master_file: string, the master file to open
    :param night: string, the night repository in form YYMMDD
    :param key: string, the key for this database entry (e.g. 'DARK')
    :param file_name: string, the filename for this database entry
    :param opt_cmt: string, the program name (for logging)
    :return None:
    """
\end{lstlisting}

\subsection{update\_master\_onlast}
\begin{lstlisting}[style=pythonstyle]
@update_master_onlast@(master_file, night, key, file_name, opt_cmt)
    """
    Update calibration database with new row in form:

    {key} {night respoitory} {filename} {human readable time} {unix time}
    
    Note calibration database is locked while this function is active
    Note time comes from file_name so must have header 'ACQTIME1'

    :param master_file: string, the master file to open
    :param night: string, the night repository in form YYMMDD
    :param key: string, the key for this database entry (e.g. 'DARK')
    :param file_name: string, the filename for this database entry
    :param opt_cmt: string, the program name (for logging)
    :return None:
    """
\end{lstlisting}

\noindent Note: exactly the same as update\_master? \\

\subsection{get\_master}
\begin{lstlisting}[style=pythonstyle]
@get_master@(master_file, max_time, opt_cmt)
    """
    Gets the latest entries (less than max_time) for all unique keys 
        (if multiple keys exist latest less than max_time is used)
        
    calibration database must be in the form
    
    {key} {night respoitory} {filename} {human readable time} {unix time}
    
    :param master_file: string, the calibration database file
    :param max_time: float, the unix time to use as the latest date than can be
                     used for calibration files
    :param opt_cmt: string, the program name (for logging)
    :return C_db: dictionary, dictionary database containing key value pairs
        
                where the C_db[key] = [night repository, calibration filename]
    """
\end{lstlisting}

\subsection{get\_early\_last\_master}
\begin{lstlisting}[style=pythonstyle]
@get_early_last_master@(master_file, opt_cmt)
    """
    Gets the earliest and latest entries for all unique keys
        (if multiple keys exist earliest and latest are used)

    calibration database must be in the form

    {key} {night respoitory} {filename} {human readable time} {unix time}

    :param master_file: string, the calibration database file
    :param opt_cmt: string, the program name (for logging)
    
    :return C_db_early: dictionary, dictionary database containing key value
                        pairs for the earliest entry of each key
    :return C_db_last: dictionary, dictionary database containing key value
                        pairs for the latest entry of each key

                where the C_db[key] = [night repository, calibration filename]
    """
\end{lstlisting}

\subsection{cp\_db\_files}
\begin{lstlisting}[style=pythonstyle]
@cp_db_files@(calib_db_dir, reduc_dir, db_dict, opt_cmt)
    """
    Copies the calibration file if it doesn't already exist in the
        reduced directory

    :param calib_db_dir: string, the calibration database directory
    :param reduc_dir: string, the reduced directory
    :param db_dict: dictionary, the calibration database dictionary
    :param opt_cmt: string, the program name (for logging)

    :return None:
    """
\end{lstlisting}

\subsection{put\_files}
\begin{lstlisting}[style=pythonstyle]
@put_file@(calib_db_dir, file, opt_cmt)
    """
    Put a calibration file in the calibration database directory
    
    :param calib_db_dir: string, the calibration directory 
    :param file: string, the calibration file to put in calibration directory
    :param opt_cmt: string, the program name (for logging)
    
    :return None: 
    """
\end{lstlisting}

% -----------------------------------------------------------------------------------
\clearpage
\newpage
\section{The spirouEXTOR module}
% -----------------------------------------------------------------------------------

Contains function for the ordres extraction

\begin{lstlisting}[style=pythonstyle]
readkeyloco(fitsfilename, kw, nbo, nbc)

@cosmic_filter@(spe, window_size=128, sigma_clip=4.5)

@extract0@(data, pos, sig, plage, ccdgain)

@extract_weigth@(data, pos, sig, plage, order_profil, ccdgain, sigdet)

@extract_tilt_weigth@(data, pos, sig, tilt, plage, order_profil, ccdgain,
                    sigdet)

@extract_weigth2@(data, pos, sig, plage1, plage2, order_profil, ccdgain,
                sigdet)

@extract_tilt_weigth2@(data, pos, sig, tilt, plage1, plage2, order_profil,
                     ccdgain, sigdet)

@extract_horne@(data, pos, sig, plage, order_profil, ccdgain, sigdet)

@extract_tilt_horne@(data, pos, sig, tilt, plage, order_profil, ccdgain,
                   sigdet)

@extract_tilt@(data, pos, sig, tilt, plage, ccdgain)

@extract@(data, pos, sig, opt, nbsig, ccdgain, sigdet, seuilcosmic)
\end{lstlisting}

% -----------------------------------------------------------------------------------
\clearpage
\newpage
\section{The spirouFITS module}
% -----------------------------------------------------------------------------------

Contains functions to read FITS file and copy keywords

\subsection{read\_data}
\begin{lstlisting}[style=pythonstyle]
@read_data@(fitsfilename)
    """
    Reads "fitsfilename" using pyfits and returns array containing the data
    only uses data from the first header

    :param fitsfilename: string, file location and file name

    :return data: numpy array (2D), the image
    :return nx: int, the shape in the first dimension, i.e. data.shape[0]
    :return ny: int, the shape in the second dimension, i.e. data.shape[1]
    """
\end{lstlisting}

\noindent Note: currently not used?

\subsection{read\_data\_raw}
\begin{lstlisting}[style=pythonstyle]
@read_data_raw@(fitsfilename)
    """
    Reads "fitsfilename" using pyfits and returns array containing the data

    If there is one extension it is used
    If there are two extensions the second is used

    :param fitsfilename: string, file location and file name

    :return data: numpy array (2D), the image
    :return nx: int, the shape in the first dimension, i.e. data.shape[0]
    :return ny: int, the shape in the second dimension, i.e. data.shape[1]
    """
\end{lstlisting}

\noindent Note: Essentially the same as `read\_data' function

\subsection{read\_data\_all}
\begin{lstlisting}[style=pythonstyle]
@read_data_all@(fitsfilename)
    """
    Reads "fitsfilename" using pyfits and returns array containing the data

    If there is one extension it is used
    If there are two extensions the second is used
    If there are multiple extensions (1 or 2 extensions) the are joined in form:

    x = [[3, 3, 3],
         [3, 3, 3],
         [3, 3, 3]])

    y = [[5, 5, 5],
         [5, 5, 5],
         [5, 5, 5]])

    returned array is = [[3, 3, 3, 5, 5, 5],
                         [3, 3, 3, 5, 5, 5],
                         [3, 3, 3, 5, 5, 5]])

    :param fitsfilename: string, file location and file name

    :return data: numpy array (2D), the image
    :return nx: int, the shape in the first dimension, i.e. data.shape[0]
    :return ny: int, the shape in the second dimension, i.e. data.shape[1]
    """

\end{lstlisting}

\subsection{write\_data}
\begin{lstlisting}[style=pythonstyle]
@write_data@(fitsfilename, data)
    """
    Create the FITS file and write data

    :param filename: string, filename to save the fits file to
    :param data: numpy array (2D), the image

    :return None:
    """
\end{lstlisting}

\subsection{read\_ext}
\begin{lstlisting}[style=pythonstyle]
@read_ext@(fitsfilename)
    """
    Opens the FITS file and reads the number of extensions

    :param fitsfilename: string, filename to read
    :return length: the length of the extention -1
    """
\end{lstlisting}

\subsection{read\_key}
\begin{lstlisting}[style=pythonstyle]
@read_key@(fitsfilename, keyname, hdu=0)
    """
    Reads the "keyname" from the header of "fitsfilename" and returns
    the value of the "keyname"

    :param fitsfilename: string, filename to read
    :param keyname: string, the keyword to look for in the header,
                    returns 'UNKNOWN' if not found
    :param hdu: int, the extention to read the header from (default = 0)

    :return None:
    """
\end{lstlisting}

\subsection{read\_keys}
\begin{lstlisting}[style=pythonstyle]
@read_keys@(fitsfilename, keylist, hdu=0)
    """
    Reads the "keyname" from the header of "fitsfilename" and returns
    the value of the "keyname"

    :param fitsfilename: string, filename to read
    :param keylist: string, the keyword to look for in the header,
                    returns 'UNKNOWN' if not found
    :param hdu: int, the extention to read the header from (default = 0)

    :return None:
\end{lstlisting}

\noindent Note: currently identical to `read\_key'

\subsection{write\_newkey}
\begin{lstlisting}[style=pythonstyle]
@write_newkey@(fitsfilename, key)
    """
    Writes a new keyword "key[0]" to the FITS file header (extension = 0)
    :param fitsfilename: string, filename to write to
    :param key: list of string, the key to write, must be in form:

            key = [KEYWORD NAME, VALUE, COMMENT]

    :return None:
    """
\end{lstlisting}

\subsection{update\_key}
\begin{lstlisting}[style=pythonstyle]
@update_key@(fitsfilename, key)
    """
    Modifies a keyword "key[0]" in the FITS file header (extension = 0)
    
    :param fitsfilename: string, filename to read/write to
    :param key: string, the key to write, must be in form:

            key = [KEYWORD NAME, VALUE, COMMENT]

    :return None :
    """
\end{lstlisting}

\subsection{delete\_key}
\begin{lstlisting}[style=pythonstyle]
@delete_key@(fitsfilename, keyname)
    """
    Deletes a keyword "keyname" in the FITS file header (extension = 0)

    :param fitsfilename: string, filename to read/write to
    :param keyname: string, the key to write, must be in form:

    :return None:
    """
\end{lstlisting}

\subsection{writekey2dlist}
\begin{lstlisting}[style=pythonstyle]
@writekey2dlist@(fitsfilename, a, kw)
    """
    Writes a 2D list of values (in "a") to FITs file "fitsfilename" with
    keyword = "kw" + number

    where number = (row number * number of columns) + column number

    :param fitsfilename: string, filename to read/write to
    :param a: 2D list of values, the values to write
    :param kw: string, the prefix of the key to write, key will be in form:
                "kw" + number

    :return None:
    """
\end{lstlisting}

\subsection{copy\_key}
\begin{lstlisting}[style=pythonstyle]
@copy_key@(fits1, fits2)
    """
    Writes the header keys from one FITS fits (fits1) to another (fits2)
    
    Does not include FITS keywords:
    
            'SIMPLE', 'BITPIX', 'NAXIS', 'NAXIS1', 'NAXIS2',
            'EXTEND', 'COMMENT', 'CRVAL1', 'CRPIX1', 'CDELT1',
            'CRVAL2', 'CRPIX2', 'CDELT2', 'BSCALE', 'BZERO',
            'PHOT_IM', 'FRAC_OBJ', 'FRAC_SKY', 'FRAC_BB'
    
    :param fits1: string, filename to read from 
    :param fits2: string, filename to write to
    
    :return None: 
    """
\end{lstlisting}

\subsection{copy\_keys\_root}
\begin{lstlisting}[style=pythonstyle]
@copy_keys_root@(fits1, fits2, root)
    """
    Copies all keywords with prefix "root" from "fits1" to "fits2"
    
    :param fits1: string, filename to read from
    :param fits2: string, filename to write to
    :param root: string, prefix - start of the keyword to read/write to
    
    :return None: 
    """
\end{lstlisting}

\subsection{readkeyloco}
\begin{lstlisting}[style=pythonstyle]
@readkeyloco@(fitsfilename, kw, nbo, nbc)
    """
    Read a key from generated from a 2D list

    :param fitsfilename: string, filename to read/write to
    :param kw: string, the prefix of the key to write, key will be in form:
                "kw" + number
                
                where number = (row number * number of columns) + column number
                
    :param nbo: int, the row number
    :param nbc: int, the columns number
    
    :return None:
    """
\end{lstlisting}

% -----------------------------------------------------------------------------------
\clearpage
\newpage
\section{The spirouLOCOR module}
% -----------------------------------------------------------------------------------

Contains functions for order localization

\subsection{meas\_back}
\begin{lstlisting}[style=pythonstyle]
@meas_back@(y, backini, niter=3, coeff=9)
\end{lstlisting}

\subsection{meas\_top}
\begin{lstlisting}[style=pythonstyle]
@meas_top@(yc, nbslice, nbpix)
\end{lstlisting}

\subsection{meas\_minmax}
\begin{lstlisting}[style=pythonstyle]
@meas_minmax@(y, size)
    """
    Measure the minimum and maximum pixel value for each row using a box which
    contains all pixels for rows:  row-size to row+size and all columns.

    Edge pixels (0-->size and (image.shape[0]-size)-->image.shape[0] are
    set to the values for row=size and row=(image.shape[0]-size)

    :param y: numpy array (2D), the image
    :param size: int, the half size of the box to use (half height)
                 so box is defined from  row-size to row+size

    :return min_image: numpy array (1D length = image.shape[0]), the row values
                       for minimum pixel defined by a box of row-size to
                       row+size for all columns
    :retrun max_image: numpy array (1D length = image.shape[0]), the row values
                       for maximum pixel defined by a box of row-size to
                       row+size for all columns
    """
\end{lstlisting}

\subsection{meas\_min}
\begin{lstlisting}[style=pythonstyle]
@meas_min@(yc, nbslice, nbpix)
\end{lstlisting}

\subsection{poscolc}
\begin{lstlisting}[style=pythonstyle]
@poscolc@(ycc, seuil)
    """
    Takes the central pixel values and finds orders by looking for the start
    and end above threshold
    
    :param ycc: numpy array (1D) size = number of rows, 
                the central pixel values
    :param seuil: float, the threshold above which to find pixels as being
                  part of an order
                      
    :return positions: numpy array (1D), size= number of rows,
                       the pixel positions in cvalues where the centers of each
                       order should be 
    """
\end{lstlisting}

\subsection{findord}
\begin{lstlisting}[style=pythonstyle]
@findord@(ycc, seuil, widthmin=5)
\end{lstlisting}

\subsection{locgaus}
\begin{lstlisting}[style=pythonstyle]
@locgaus@(y, ind1, ind2, sigdet, opt=1)
\end{lstlisting}

\subsection{loc2gaus}
\begin{lstlisting}[style=pythonstyle]
@loc2gaus@(y, ind1, ind2, delta, sigdet, opt=1)
\end{lstlisting}

\subsection{fitprofil}
\begin{lstlisting}[style=pythonstyle]
@fitprofil@(y, sigdet)
\end{lstlisting}

\subsection{imaloco}
\begin{lstlisting}[style=pythonstyle]
@imaloco@(data, ac, fitsfilename)
\end{lstlisting}

\subsection{imaloco2}
\begin{lstlisting}[style=pythonstyle]
@imaloco2@(data, ac, fitsfilename)
\end{lstlisting}

\subsection{tableloco}
\begin{lstlisting}[style=pythonstyle]
@tableloco@(tblfilename, a)
\end{lstlisting}

\subsection{keyloco}
\begin{lstlisting}[style=pythonstyle]
@keyloco@(fitsfilename, ac, ass)
\end{lstlisting}

\subsection{keyloco2}
\begin{lstlisting}[style=pythonstyle]
@keyloco2@(fitsfilename, ac, ass)
\end{lstlisting}

\subsection{writekeyloco3}
\begin{lstlisting}[style=pythonstyle]
@writekeyloco3@(fitsfilename, a)
\end{lstlisting}

\subsection{writekeyloco}
\begin{lstlisting}[style=pythonstyle]
@writekeyloco@(fitsfilename, a, kw)
\end{lstlisting}

\subsection{readkeyloco}
\begin{lstlisting}[style=pythonstyle]
@readkeyloco@(fitsfilename, ic_locnbmaxo, ic_locdfitc)
\end{lstlisting}

\subsection{readkeyloco3}
\begin{lstlisting}[style=pythonstyle]
@readkeyloco3@(fitsfilename, rootkeyname)
\end{lstlisting}

\subsection{write\_fitsloco}
\begin{lstlisting}[style=pythonstyle]
@write_fitsloco@(fitsfilename, a, dim)
\end{lstlisting}


\clearpage
\newpage
\section{The spirouRV module}
% -----------------------------------------------------------------------------------

Contains functions to compute radial velocity

% -----------------------------------------------------------------------------------
\clearpage
\newpage
\section{The spirouVISU module}
% -----------------------------------------------------------------------------------

Contains functions for visualization
