% ---------------------------------------------------------------
\chapter{The Modules}
\label{section:the_modules}
% ---------------------------------------------------------------

Some text here

% ---------------------------------------------------------------
\section{The Module directory}
% ---------------------------------------------------------------

This contains all the python worker files for the DRS. The file structure is as below.

\customdirtree{%
.1 \{DIR\}.
.2 \{INSTALL\_ROOT\}.
.3 DRS\_SPIROU.
.4 python.
.5 Modules.
.6 hadgtVISU.
.7 \DTcomment{hadmrVISU module files}.
.6 hadmrBACK.
.7 \DTcomment{hadmrBACK module files}.
.6 hadmrBIAS.
.7 \DTcomment{hadmrBIAS module files}.
.6 hadmrCDB.
.7 \DTcomment{hadmrCDB module files}.
.6 hadmrDARK.
.7 \DTcomment{hadmrDARK module files}.
.6 hadmrEXTOR.
.7 \DTcomment{hadmrEXTOR module files}.
.6 hadmrFITS.
.7 \DTcomment{hadmrFITS module files}.
.6 hadmrFLAT.
.7 \DTcomment{hadmrFLAT module files}.
.6 hadmrLED.
.7 \DTcomment{hadmrLED module files}.
.6 hadmrLOCOR.
.7 \DTcomment{hadmrLOCOR module files}.
.6 hadmrMATH.
.7 \DTcomment{hadmrMATH module files}.
.6 hadmrRV.
.7 \DTcomment{hadmrRV module files}.
.6 hadmrSIMU.
.7 \DTcomment{hadmrSIMU module files}.
.6 hadmrTHORCA.
.7 \DTcomment{hadmrTHORCA module files}.
.6 hadrgdCONFIG.
.7 \DTcomment{hadrgdCONFIG module files}.
.6 spirouBACK.py.
.6 spirouBIAS.py.
.6 spirouCDB.py.
.6 spirouEXTOR.py.
.6 spirouFITS.py.
.6 spirouFLAT.py.
.6 spirouLOCOR.py.
.6 spirouRV.py.
.6 spirouTHORCA.py.
.6 spirouVISU.py.
}

% -----------------------------------------------------------------------------------
\clearpage
\newpage
\section{The spirouBACK module}
% -----------------------------------------------------------------------------------

Contains functions to calculate the detector background

\subsection{measure\_bkgr}

\begin{lstlisting}[style=pythonstyle]
@measure_bkgr@(data, order_profile, size, ccdgain, ccdsigdet)
    """
    Measures the background via interpolation over many small boxes of
         width/height="size" uses the order_profile to mask out (?) the orders
         TODO: Is that what this is doing?

    :param data: 2D numpy array, the image to measure the background of
    :param order_profile: 2D numpy array, the smoothed image using the order
                          fits
    :param size: int, size of the sub-frame to measure the background in
    :param ccdgain: float, the gain of the image
    :param ccdsigdet: float, the sigdet of the image

    :return bkgr: 2D numpy array, the interpolated background image
    :return xc: numpy array (size x data.shape[0]) in steps of 2*size
    :return yc: numpy array (size x data.shape[1]) in steps of 2*size
    :return mode: numpy array (size x size) the mode (?) of each box
    """
\end{lstlisting}

\noindent Note: Not 100\% sure how this function works. \\

\subsection{measure\_bkgr2}

\begin{lstlisting}[style=pythonstyle]
@measure_bkgr2@(data, order_profile, size, ccdgain, ccdsigdet)
    """
    Measures the background via interpolation over many small boxes of
         width/height="size" uses the order_profile to mask out (?) the orders
         TODO: Is that what this is doing?

    :param data: 2D numpy array, the image to measure the background of
    :param order_profile: 2D numpy array, the smoothed image using the order
                          fits
    :param size: int, size of the sub-frame to measure the background in
    :param ccdgain: float, the gain of the image
    :param ccdsigdet: float, the sigdet of the image

    :return bkgr: 2D numpy array, the interpolated background image
    :return xc: numpy array (size x data.shape[0]) in steps of 2*size
    :return yc: numpy array (size x data.shape[1]) in steps of 2*size
    :return mode: numpy array (size x size) the mode (?) of each box
    :return binc: numpy array, the bins the histogram is binned along
    :return histo: numpy array, the histogram of each subframe
    """
\end{lstlisting}

\noindent Note: Not 100\% sure how this function works. \\

\subsection{measure\_bkgr\_FF}

\begin{lstlisting}[style=pythonstyle]
@measure_bkgr_FF@(data, size, ccdgain, ccdsigdet)
    """
    Measures the background via interpolation over many small boxes of 
         width/height="size"
    
    :param data: 2D numpy array, the image to measure the background of
    :param size: int, size of the sub-frame to measure the background in
    :param ccdgain: float, the gain of the image
    :param ccdsigdet: float, the sigdet of the image
    
    :return bkgr: 2D numpy array, the interpolated background image
    :return xc: numpy array (size x data.shape[0]) in steps of 2*size
    :return yc: numpy array (size x data.shape[1]) in steps of 2*size
    :return minlevel: numpy array (size x size) - the central value of each
                      subframed box
    """
\end{lstlisting}


\subsection{measure\_bkgr\_FF2}

\begin{lstlisting}[style=pythonstyle]
@measure_bkgr_FF2@(data, size, ccdgain, ccdsigdet, seuil)
    """
    Measures the background via interpolation over many small boxes of 
         width/height="size" uses a threshold="seuil" above which the value is
         set to the max value in a box

    :param data: 2D numpy array, the image to measure the background of
    :param size: int, size of the sub-frame to measure the background in
    :param ccdgain: float, the gain of the image
    :param ccdsigdet: float, the sigdet of the image
    "param seuil: float, a threshold for 

    :return bkgr: 2D numpy array, the interpolated background image
    :return xc: numpy array (size x data.shape[0]) in steps of 2*size
    :return yc: numpy array (size x data.shape[1]) in steps of 2*size
    :return minlevel: numpy array (size x size) - the central value of each
                      subframed box
    """
\end{lstlisting}


% -----------------------------------------------------------------------------------
\clearpage
\newpage
\section{The spirouCDB module}
% -----------------------------------------------------------------------------------

Contains functions for the calibrations database (infos for master\_calib.txt, and to coy files in the caliDB directory)

\subsection{filename2tunix}
\begin{lstlisting}[style=pythonstyle]
@filename2tunix@(cfht_name)
    """
    Open the fits file "cfht_name" read the header key 'DATEEND' and return
    it as a unix timestamp
    
    :param cfht_name: string, the fits filename and location of file
    :return tt: float, the unix time of "cfht_name" store in 'DATEEND' 
    """
\end{lstlisting}

\noindent Note: Currently not used in code \\

\subsection{filename2tunix2}
\begin{lstlisting}[style=pythonstyle]
@filename2tunix2@(cfht_name)
    """
    Open the fits file "cfht_name" read the header key 'ACQTIME1' and return
    it as a unix timestamp

    :param cfht_name: string, the fits filename and location of file
    :return tt: float, the unix time of "cfht_name" store in 'ACQTIME1'
    """
\end{lstlisting}

\subsection{filename2tuni\_old}
\begin{lstlisting}[style=pythonstyle]
@filename2tunix_old@(file_name)
    """
    Extracts the timestamp information from a filename and converts it to a 
        unix timestamp

    :param file_name: string, filename toe look for timestamp in, must contain
                      *.YY-MM-DDThh:mm:ss_*  or *.YY-MM-DDThh:mm:ss.fits
                      i.e. filename.17-06-12T15:32:42
                      
    :return tt: float, the unix time from file_name
    """
\end{lstlisting}

\noindent Note: Currently not used in code \\

\subsection{data2tunix}
\begin{lstlisting}[style=pythonstyle]
@date2tunix@(argdate)
    """
    Converts a string date in format YY-MM-DD into a unix timestamp

    :param argdate: string in form YY-MM-DD
    
    :return tt: float, the unix time from argdate
    """
\end{lstlisting}

\noindent Note: Is this used? \\

\subsection{update\_master}
\begin{lstlisting}[style=pythonstyle]
@update_master@(master_file, night, key, file_name, opt_cmt)
    """
    Update calibration database with new row in form:

    {key} {night respoitory} {filename} {human readable time} {unix time}
    
    Note calibration database is locked while this function is active
    Note time comes from file_name so must have header 'ACQTIME1'

    :param master_file: string, the master file to open
    :param night: string, the night repository in form YYMMDD
    :param key: string, the key for this database entry (e.g. 'DARK')
    :param file_name: string, the filename for this database entry
    :param opt_cmt: string, the program name (for logging)
    :return None:
    """
\end{lstlisting}

\subsection{update\_master\_onlast}
\begin{lstlisting}[style=pythonstyle]
@update_master_onlast@(master_file, night, key, file_name, opt_cmt)
    """
    Update calibration database with new row in form:

    {key} {night respoitory} {filename} {human readable time} {unix time}
    
    Note calibration database is locked while this function is active
    Note time comes from file_name so must have header 'ACQTIME1'

    :param master_file: string, the master file to open
    :param night: string, the night repository in form YYMMDD
    :param key: string, the key for this database entry (e.g. 'DARK')
    :param file_name: string, the filename for this database entry
    :param opt_cmt: string, the program name (for logging)
    :return None:
    """
\end{lstlisting}

\noindent Note: exactly the same as update\_master? \\

\subsection{get\_master}
\begin{lstlisting}[style=pythonstyle]
@get_master@(master_file, max_time, opt_cmt)
    """
    Gets the latest entries (less than max_time) for all unique keys 
        (if multiple keys exist latest less than max_time is used)
        
    calibration database must be in the form
    
    {key} {night respoitory} {filename} {human readable time} {unix time}
    
    :param master_file: string, the calibration database file
    :param max_time: float, the unix time to use as the latest date than can be
                     used for calibration files
    :param opt_cmt: string, the program name (for logging)
    :return C_db: dictionary, dictionary database containing key value pairs
        
                where the C_db[key] = [night repository, calibration filename]
    """
\end{lstlisting}

\subsection{get\_early\_last\_master}
\begin{lstlisting}[style=pythonstyle]
@get_early_last_master@(master_file, opt_cmt)
    """
    Gets the earliest and latest entries for all unique keys
        (if multiple keys exist earliest and latest are used)

    calibration database must be in the form

    {key} {night respoitory} {filename} {human readable time} {unix time}

    :param master_file: string, the calibration database file
    :param opt_cmt: string, the program name (for logging)
    
    :return C_db_early: dictionary, dictionary database containing key value
                        pairs for the earliest entry of each key
    :return C_db_last: dictionary, dictionary database containing key value
                        pairs for the latest entry of each key

                where the C_db[key] = [night repository, calibration filename]
    """
\end{lstlisting}

\subsection{cp\_db\_files}
\begin{lstlisting}[style=pythonstyle]
@cp_db_files@(calib_db_dir, reduc_dir, db_dict, opt_cmt)
    """
    Copies the calibration file if it doesn't already exist in the
        reduced directory

    :param calib_db_dir: string, the calibration database directory
    :param reduc_dir: string, the reduced directory
    :param db_dict: dictionary, the calibration database dictionary
    :param opt_cmt: string, the program name (for logging)

    :return None:
    """
\end{lstlisting}

\subsection{put\_files}
\begin{lstlisting}[style=pythonstyle]
@put_file@(calib_db_dir, file, opt_cmt)
    """
    Put a calibration file in the calibration database directory
    
    :param calib_db_dir: string, the calibration directory 
    :param file: string, the calibration file to put in calibration directory
    :param opt_cmt: string, the program name (for logging)
    
    :return None: 
    """
\end{lstlisting}

% -----------------------------------------------------------------------------------
\clearpage
\newpage
\section{The spirouEXTOR module}
% -----------------------------------------------------------------------------------

Contains function for the ordres extraction

\begin{lstlisting}[style=pythonstyle]
readkeyloco(fitsfilename, kw, nbo, nbc)

@cosmic_filter@(spe, window_size=128, sigma_clip=4.5)

@extract0@(data, pos, sig, plage, ccdgain)

@extract_weigth@(data, pos, sig, plage, order_profil, ccdgain, sigdet)

@extract_tilt_weigth@(data, pos, sig, tilt, plage, order_profil, ccdgain,
                    sigdet)

@extract_weigth2@(data, pos, sig, plage1, plage2, order_profil, ccdgain,
                sigdet)

@extract_tilt_weigth2@(data, pos, sig, tilt, plage1, plage2, order_profil,
                     ccdgain, sigdet)

@extract_horne@(data, pos, sig, plage, order_profil, ccdgain, sigdet)

@extract_tilt_horne@(data, pos, sig, tilt, plage, order_profil, ccdgain,
                   sigdet)

@extract_tilt@(data, pos, sig, tilt, plage, ccdgain)

@extract@(data, pos, sig, opt, nbsig, ccdgain, sigdet, seuilcosmic)
\end{lstlisting}

% -----------------------------------------------------------------------------------
\clearpage
\newpage
\section{The spirouFITS module}
% -----------------------------------------------------------------------------------

Contains functions to read FITS file and copy keywords

\begin{lstlisting}[style=pythonstyle]
@read_data@(fitsfilename)

@read_data_raw@(fitsfilename)

@read_data_all@(fitsfilename)

@write_data@(fitsfilename, data)

@read_ext@(fitsfilename)

@read_key@(fitsfilename, keyname, hdu=0)

@read_keys@(fitsfilename, keylist, hdu=0)

@write_newkey@(fitsfilename, key)

@update_key@(fitsfilename, key)

@delete_key@(fitsfilename, keyname)

@writekey2dlist@(fitsfilename, a, kw)

@copy_key@(fits1, fits2)

@copy_keys_root@(fits1, fits2, root)

@readkeyloco@(fitsfilename, kw, nbo, nbc)
\end{lstlisting}

% -----------------------------------------------------------------------------------
\clearpage
\newpage
\section{The spirouLOCOR module}
% -----------------------------------------------------------------------------------

Contains functions for order localization

\begin{lstlisting}[style=pythonstyle]
@meas_back@(y, backini, niter=3, coeff=9)

@meas_top@(yc, nbslice, nbpix)

@meas_minmax@(y, size)

@meas_min@(yc, nbslice, nbpix)

@poscolc@(ycc, seuil)

@findord@(ycc, seuil, widthmin=5)

@locgaus@(y, ind1, ind2, sigdet, opt=1)

@loc2gaus@(y, ind1, ind2, delta, sigdet, opt=1)

@fitprofil@(y, sigdet)

@imaloco@(data, ac, fitsfilename)

@imaloco2@(data, ac, fitsfilename)

@tableloco@(tblfilename, a)

@keyloco@(fitsfilename, ac, ass)

@keyloco2@(fitsfilename, ac, ass)

@writekeyloco3@(fitsfilename, a)

@writekeyloco@(fitsfilename, a, kw)

@readkeyloco@(fitsfilename, ic_locnbmaxo, ic_locdfitc)

@readkeyloco3@(fitsfilename, rootkeyname)

@write_fitsloco@(fitsfilename, a, dim)
\end{lstlisting}


\clearpage
\newpage
\section{The spirouRV module}
% -----------------------------------------------------------------------------------

Contains functions to compute radial velocity

% -----------------------------------------------------------------------------------
\clearpage
\newpage
\section{The spirouVISU module}
% -----------------------------------------------------------------------------------

Contains functions for visualization
