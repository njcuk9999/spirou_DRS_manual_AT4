% ---------------------------------------------------------------
\chapter{The Modules}
\label{section:the_modules}
% ---------------------------------------------------------------

Some text here

% ---------------------------------------------------------------
\section{The Module directory}
% ---------------------------------------------------------------

This contains all the python worker files for the DRS. The file structure is as below.

\customdirtree{%
.1 \{DIR\}.
.2 \{INSTALL\_ROOT\}.
.3 DRS\_SPIROU.
.4 python.
.5 Modules.
.6 hadgtVISU.
.7 \DTcomment{hadmrVISU module files}.
.6 hadmrBACK.
.7 \DTcomment{hadmrBACK module files}.
.6 hadmrBIAS.
.7 \DTcomment{hadmrBIAS module files}.
.6 hadmrCDB.
.7 \DTcomment{hadmrCDB module files}.
.6 hadmrDARK.
.7 \DTcomment{hadmrDARK module files}.
.6 hadmrEXTOR.
.7 \DTcomment{hadmrEXTOR module files}.
.6 hadmrFITS.
.7 \DTcomment{hadmrFITS module files}.
.6 hadmrFLAT.
.7 \DTcomment{hadmrFLAT module files}.
.6 hadmrLED.
.7 \DTcomment{hadmrLED module files}.
.6 hadmrLOCOR.
.7 \DTcomment{hadmrLOCOR module files}.
.6 hadmrMATH.
.7 \DTcomment{hadmrMATH module files}.
.6 hadmrRV.
.7 \DTcomment{hadmrRV module files}.
.6 hadmrSIMU.
.7 \DTcomment{hadmrSIMU module files}.
.6 hadmrTHORCA.
.7 \DTcomment{hadmrTHORCA module files}.
.6 hadrgdCONFIG.
.7 \DTcomment{hadrgdCONFIG module files}.
.6 spirouBACK.py.
.6 spirouBIAS.py.
.6 spirouCDB.py.
.6 spirouEXTOR.py.
.6 spirouFITS.py.
.6 spirouFLAT.py.
.6 spirouLOCOR.py.
.6 spirouRV.py.
.6 spirouTHORCA.py.
.6 spirouVISU.py.
}

% -----------------------------------------------------------------------------------
\clearpage
\newpage
\section{The spirouBACK module}
% -----------------------------------------------------------------------------------

Contains functions to calculate the detector background

\subsection{measure\_bkgr}

\begin{lstlisting}[style=pythonstyle]
@measure_bkgr@(data, order_profile, size, ccdgain, ccdsigdet)
    """
    Measures the background via interpolation over many small boxes of
         width/height="size" uses the order_profile to mask out (?) the orders

    :param data: 2D numpy array, the image to measure the background of
    :param order_profile: 2D numpy array, the smoothed image using the order
                          fits
    :param size: int, size of the sub-frame to measure the background in
    :param ccdgain: float, the gain of the image
    :param ccdsigdet: float, the sigdet of the image

    :return bkgr: 2D numpy array, the interpolated background image
    :return xc: numpy array (size x data.shape[0]) in steps of 2*size
    :return yc: numpy array (size x data.shape[1]) in steps of 2*size
    :return mode: numpy array (size x size) the mode (?) of each box
    """
\end{lstlisting}

\noindent Note: Not 100\% sure how this function works. \\

\subsection{measure\_bkgr2}

\begin{lstlisting}[style=pythonstyle]
@measure_bkgr2@(data, order_profile, size, ccdgain, ccdsigdet)
    """
    Measures the background via interpolation over many small boxes of
         width/height="size" uses the order_profile to mask out (?) the orders

    :param data: 2D numpy array, the image to measure the background of
    :param order_profile: 2D numpy array, the smoothed image using the order
                          fits
    :param size: int, size of the sub-frame to measure the background in
    :param ccdgain: float, the gain of the image
    :param ccdsigdet: float, the sigdet of the image

    :return bkgr: 2D numpy array, the interpolated background image
    :return xc: numpy array (size x data.shape[0]) in steps of 2*size
    :return yc: numpy array (size x data.shape[1]) in steps of 2*size
    :return mode: numpy array (size x size) the mode (?) of each box
    :return binc: numpy array, the bins the histogram is binned along
    :return histo: numpy array, the histogram of each subframe
    """
\end{lstlisting}

\noindent Note: Not 100\% sure how this function works. \\

\subsection{measure\_bkgr\_FF}

\begin{lstlisting}[style=pythonstyle]
@measure_bkgr_FF@(data, size, ccdgain, ccdsigdet)
    """
    Measures the background via interpolation over many small boxes of 
         width/height="size"
    
    :param data: 2D numpy array, the image to measure the background of
    :param size: int, size of the sub-frame to measure the background in
    :param ccdgain: float, the gain of the image
    :param ccdsigdet: float, the sigdet of the image
    
    :return bkgr: 2D numpy array, the interpolated background image
    :return xc: numpy array (size x data.shape[0]) in steps of 2*size
    :return yc: numpy array (size x data.shape[1]) in steps of 2*size
    :return minlevel: numpy array (size x size) - the central value of each
                      subframed box
    """
\end{lstlisting}


\subsection{measure\_bkgr\_FF2}

\begin{lstlisting}[style=pythonstyle]
@measure_bkgr_FF2@(data, size, ccdgain, ccdsigdet, seuil)
    """
    Measures the background via interpolation over many small boxes of 
         width/height="size" uses a threshold="seuil" above which the value is
         set to the max value in a box

    :param data: 2D numpy array, the image to measure the background of
    :param size: int, size of the sub-frame to measure the background in
    :param ccdgain: float, the gain of the image
    :param ccdsigdet: float, the sigdet of the image
    "param seuil: float, a threshold for 

    :return bkgr: 2D numpy array, the interpolated background image
    :return xc: numpy array (size x data.shape[0]) in steps of 2*size
    :return yc: numpy array (size x data.shape[1]) in steps of 2*size
    :return minlevel: numpy array (size x size) - the central value of each
                      subframed box
    """
\end{lstlisting}


% -----------------------------------------------------------------------------------
\clearpage
\newpage
\section{The spirouCDB module}
% -----------------------------------------------------------------------------------

Contains functions for the calibrations database (infos for master\_calib.txt, and to coy files in the caliDB directory)

\subsection{filename2tunix}
\begin{lstlisting}[style=pythonstyle]
@filename2tunix@(cfht_name)
    """
    Open the fits file "cfht_name" read the header key 'DATEEND' and return
    it as a unix timestamp
    
    :param cfht_name: string, the fits filename and location of file
    :return tt: float, the unix time of "cfht_name" store in 'DATEEND' 
    """
\end{lstlisting}

\noindent Note: Currently not used in code \\

\subsection{filename2tunix2}
\begin{lstlisting}[style=pythonstyle]
@filename2tunix2@(cfht_name)
    """
    Open the fits file "cfht_name" read the header key 'ACQTIME1' and return
    it as a unix timestamp

    :param cfht_name: string, the fits filename and location of file
    :return tt: float, the unix time of "cfht_name" store in 'ACQTIME1'
    """
\end{lstlisting}

\vspace{0.5cm}
\subsection{filename2tuni\_old}
\begin{lstlisting}[style=pythonstyle]
@filename2tunix_old@(file_name)
    """
    Extracts the timestamp information from a filename and converts it to a 
        unix timestamp

    :param file_name: string, filename toe look for timestamp in, must contain
                      *.YY-MM-DDThh:mm:ss_*  or *.YY-MM-DDThh:mm:ss.fits
                      i.e. filename.17-06-12T15:32:42
                      
    :return tt: float, the unix time from file_name
    """
\end{lstlisting}

\noindent Note: Currently not used in code \\

\subsection{data2tunix}
\begin{lstlisting}[style=pythonstyle]
@date2tunix@(argdate)
    """
    Converts a string date in format YY-MM-DD into a unix timestamp

    :param argdate: string in form YY-MM-DD
    
    :return tt: float, the unix time from argdate
    """
\end{lstlisting}

\noindent Note: Is this used? \\

\subsection{update\_master}
\begin{lstlisting}[style=pythonstyle]
@update_master@(master_file, night, key, file_name, opt_cmt)
    """
    Update calibration database with new row in form:

    {key} {night respoitory} {filename} {human readable time} {unix time}
    
    Note calibration database is locked while this function is active
    Note time comes from file_name so must have header 'ACQTIME1'

    :param master_file: string, the master file to open
    :param night: string, the night repository in form YYMMDD
    :param key: string, the key for this database entry (e.g. 'DARK')
    :param file_name: string, the filename for this database entry
    :param opt_cmt: string, the program name (for logging)
    :return None:
    """
\end{lstlisting}

\vspace{0.5cm}
\subsection{update\_master\_onlast}
\begin{lstlisting}[style=pythonstyle]
@update_master_onlast@(master_file, night, key, file_name, opt_cmt)
    """
    Update calibration database with new row in form:

    {key} {night respoitory} {filename} {human readable time} {unix time}
    
    Note calibration database is locked while this function is active
    Note time comes from file_name so must have header 'ACQTIME1'

    :param master_file: string, the master file to open
    :param night: string, the night repository in form YYMMDD
    :param key: string, the key for this database entry (e.g. 'DARK')
    :param file_name: string, the filename for this database entry
    :param opt_cmt: string, the program name (for logging)
    :return None:
    """
\end{lstlisting}

\noindent Note: exactly the same as update\_master? \\

\subsection{get\_master}
\begin{lstlisting}[style=pythonstyle]
@get_master@(master_file, max_time, opt_cmt)
    """
    Gets the latest entries (less than max_time) for all unique keys 
        (if multiple keys exist latest less than max_time is used)
        
    calibration database must be in the form
    
    {key} {night respoitory} {filename} {human readable time} {unix time}
    
    :param master_file: string, the calibration database file
    :param max_time: float, the unix time to use as the latest date than can be
                     used for calibration files
    :param opt_cmt: string, the program name (for logging)
    :return C_db: dictionary, dictionary database containing key value pairs
        
                where the C_db[key] = [night repository, calibration filename]
    """
\end{lstlisting}

\vspace{0.5cm}
\subsection{get\_early\_last\_master}
\begin{lstlisting}[style=pythonstyle]
@get_early_last_master@(master_file, opt_cmt)
    """
    Gets the earliest and latest entries for all unique keys
        (if multiple keys exist earliest and latest are used)

    calibration database must be in the form

    {key} {night respoitory} {filename} {human readable time} {unix time}

    :param master_file: string, the calibration database file
    :param opt_cmt: string, the program name (for logging)
    
    :return C_db_early: dictionary, dictionary database containing key value
                        pairs for the earliest entry of each key
    :return C_db_last: dictionary, dictionary database containing key value
                        pairs for the latest entry of each key

                where the C_db[key] = [night repository, calibration filename]
    """
\end{lstlisting}

\vspace{0.5cm}
\subsection{cp\_db\_files}
\begin{lstlisting}[style=pythonstyle]
@cp_db_files@(calib_db_dir, reduc_dir, db_dict, opt_cmt)
    """
    Copies the calibration file if it doesn't already exist in the
        reduced directory

    :param calib_db_dir: string, the calibration database directory
    :param reduc_dir: string, the reduced directory
    :param db_dict: dictionary, the calibration database dictionary
    :param opt_cmt: string, the program name (for logging)

    :return None:
    """
\end{lstlisting}

\vspace{0.5cm}
\subsection{put\_files}
\begin{lstlisting}[style=pythonstyle]
@put_file@(calib_db_dir, file, opt_cmt)
    """
    Put a calibration file in the calibration database directory
    
    :param calib_db_dir: string, the calibration directory 
    :param file: string, the calibration file to put in calibration directory
    :param opt_cmt: string, the program name (for logging)
    
    :return None: 
    """
\end{lstlisting}

% -----------------------------------------------------------------------------------
\clearpage
\newpage
\section{The spirouEXTOR module}
% -----------------------------------------------------------------------------------

Contains function for the ordres extraction

\subsection{readkeyloco}
\begin{lstlisting}[style=pythonstyle]
readkeyloco(fitsfilename, kw, nbo, nbc)
    """
    Reads the localisation parameters from fitsfilename header

    :param fitsfilename: string, the FITS file name to read
    :param kw: string, the prefix for the FITS file header constant
    :param nbo: int, the number of orders to extract from header
    :param nbc: int, the number of fit coefficients for each order
   
    :return value: numpy array (2D), the values from the header
                   size = nbo x nbc
    """
\end{lstlisting}

\subsection{cosmic\_filter}
\begin{lstlisting}[style=pythonstyle]
@cosmic_filter@(spe, window_size=128, sigma_clip=4.5)
    """
    Corrects for cosmic rays via a positive sigma clipping on a window
    
    :param spe: the extracted image for this order
    :param window_size: int, the size of the window
    :param sigma_clip: float, the sigma of the sigma clip
    
    :return spec: the corrected extracted image for this order
    :return nbcos: 0 
    """

\end{lstlisting}

\subsection{extract0}
\begin{lstlisting}[style=pythonstyle]
@extract0@(data, pos, sig, plage, ccdgain)
    """
    Extraction of this order using summation over a constant range
    
    :param data: numpy array (2D), the image to be extracted
    :param pos: numpy array (1D), the fit coefficients for the position 
                centers for this order
    :param sig: numpy array (1D), the fit coefficients for the widths for this
                order
    :param plage: float, the size in number of rows around the centers to fit
                  to (y-axis dimension)
    :param ccdgain: float, the gain of the image
    
    :return: numpy array (1D), the extracted pixels for this order
    """
\end{lstlisting}

\subsection{extract\_weigth}
\begin{lstlisting}[style=pythonstyle]
@extract_weigth@(data, pos, sig, plage, order_profil, ccdgain, sigdet)
    """
    Extraction of Horne with weights from the order profile
    
    :param data: numpy array (2D), the image to be extracted
    :param pos: numpy array (1D), the fit coefficients for the position
                centers for this order
    :param sig: numpy array (1D), the fit coefficients for the widths for this
                order
    :param plage: float, the size in number of rows around the centers to fit
                  to (y-axis dimension)
    :param order_profil: numpy array (2D), the order profile of the image
    :param ccdgain: float, the gain of the image
    :param sigdet: float, the sig det for the image
    
    :return: numpy array (1D), the extracted pixels for this order
    """
\end{lstlisting}

\subsection{extract\_tilt\_weigth}
\begin{lstlisting}[style=pythonstyle]
@extract_tilt_weigth@(data, pos, sig, tilt, plage, order_profil, ccdgain,
                    sigdet)
    """
    Extraction of this order with tilt and weights from order profile
    
    :param data: numpy array (2D), the image to be extracted
    :param pos: numpy array (1D), the fit coefficients for the position
                centers for this order
    :param sig: numpy array (1D), the fit coefficients for the widths for this
                order
    :param tilt: float, the tilt for this order (from cal_SLIT via calibDB)
    :param plage: float, the size in number of rows around the centers to fit
                  to (y-axis dimension)
    :param order_profil: numpy array (2D), the order profile of the image
    :param ccdgain: float, the gain of the image
    :param sigdet: float, the sig det for the image

    :return: numpy array (1D), the extracted pixels for this order
    """
\end{lstlisting}

\subsection{extract\_weigth2}
\begin{lstlisting}[style=pythonstyle]
@extract_weigth2@(data, pos, sig, plage1, plage2, order_profil, ccdgain,
                sigdet)
    """
    Extraction of Horne with weights from the order profile, with option for
    unequal range for the width around the central position

    :param data: numpy array (2D), the image to be extracted
    :param pos: numpy array (1D), the fit coefficients for the position
                centers for this order
    :param sig: numpy array (1D), the fit coefficients for the widths for this
                order
    :param plage1: float, the size in number of rows around the centers to fit
                  to (y-axis dimension) for the rows before (top y-dimension)
    :param plage2: float, the size in number of rows around the centers to fit
                  to (y-axis dimension) for the rows after (bottom y-dimension)
    :param order_profil: numpy array (2D), the order profile of the image
    :param ccdgain: float, the gain of the image
    :param sigdet: float, the sig det for the image

    :return: numpy array (1D), the extracted pixels for this order
    """
\end{lstlisting}

\subsection{extract\_tilt\_weigth2}
\begin{lstlisting}[style=pythonstyle]
@extract_tilt_weigth2@(data, pos, sig, tilt, plage1, plage2, order_profil,
                     ccdgain, sigdet)
    """
    Extraction of this order with tilt and weights from order profile, with 
    option for unequal range for the width around the central position

    :param data: numpy array (2D), the image to be extracted
    :param pos: numpy array (1D), the fit coefficients for the position
                centers for this order
    :param sig: numpy array (1D), the fit coefficients for the widths for this
                order
    :param tilt: float, the tilt for this order (from cal_SLIT via calibDB)
    :param plage1: float, the size in number of rows around the centers to fit
                  to (y-axis dimension) for the rows before (top y-dimension)
    :param plage2: float, the size in number of rows around the centers to fit
                  to (y-axis dimension) for the rows after (bottom y-dimension)
    :param order_profil: numpy array (2D), the order profile of the image
    :param ccdgain: float, the gain of the image
    :param sigdet: float, the sig det for the image

    :return: numpy array (1D), the extracted pixels for this order
    """
\end{lstlisting}

\subsection{extract\_horne}
\begin{lstlisting}[style=pythonstyle]
@extract_horne@(data, pos, sig, plage, order_profil, ccdgain, sigdet)
    """
    Extraction of Horne with weights from the order profile using the variance
    as a normalisation

    :param data: numpy array (2D), the image to be extracted
    :param pos: numpy array (1D), the fit coefficients for the position
                centers for this order
    :param sig: numpy array (1D), the fit coefficients for the widths for this
                order
    :param plage: float, the size in number of rows around the centers to fit
                  to (y-axis dimension)
    :param order_profil: numpy array (2D), the order profile of the image
    :param ccdgain: float, the gain of the image
    :param sigdet: float, the sig det for the image

    :return: numpy array (1D), the extracted pixels for this order
    """

\end{lstlisting}

\subsection{extract\_tilt\_horne}
\begin{lstlisting}[style=pythonstyle]
@extract_tilt_horne@(data, pos, sig, tilt, plage, order_profil, ccdgain,
                   sigdet)
    """
    Extraction of this order with tilt from order profile using the variance
    as a normalisation

    :param data: numpy array (2D), the image to be extracted
    :param pos: numpy array (1D), the fit coefficients for the position
                centers for this order
    :param sig: numpy array (1D), the fit coefficients for the widths for this
                order
    :param tilt: float, the tilt for this order (from cal_SLIT via calibDB)
    :param plage: float, the size in number of rows around the centers to fit
                  to (y-axis dimension)
    :param order_profil: numpy array (2D), the order profile of the image
    :param ccdgain: float, the gain of the image
    :param sigdet: float, the sig det for the image

    :return: numpy array (1D), the extracted pixels for this order
    """
\end{lstlisting}

\subsection{extract\_tilt}
\begin{lstlisting}[style=pythonstyle]
@extract_tilt@(data, pos, sig, tilt, plage, ccdgain)
    """
    Extraction of this order with tilt from order profile

    :param data: numpy array (2D), the image to be extracted
    :param pos: numpy array (1D), the fit coefficients for the position
                centers for this order
    :param sig: numpy array (1D), the fit coefficients for the widths for this
                order
    :param tilt: float, the tilt for this order (from cal_SLIT via calibDB)
    :param plage: float, the size in number of rows around the centers to fit
                  to (y-axis dimension)
    :param ccdgain: float, the gain of the image

    :return: numpy array (1D), the extracted pixels for this order
    """
\end{lstlisting}

\subsection{extract}
\begin{lstlisting}[style=pythonstyle]
@extract@(data, pos, sig, opt, nbsig, ccdgain, sigdet, seuilcosmic)
    """
    Extraction function with multiple options set by opt:
        0 = extraction over a constant range
        1 = extraction from summation over a constant sigma
        2 = horne extraction
        3 = horce extraction with cosmic elimination
    
    :param data: numpy array (2D), the image to be extracted
    :param pos: numpy array (1D), the fit coefficients for the position
                centers for this order
    :param sig: numpy array (1D), the fit coefficients for the widths for this
                order
    :param opt: int, defines extraction option:
        0 = extraction over a constant range
        1 = extraction from summation over a constant sigma
        2 = horne extraction
        3 = horce extraction with cosmic elimination
    
    :param nbsig: float, the size in number of rows around the centers to fit
                  to (y-axis dimension)
    :param ccdgain: float, the gain of the image
    :param sigdet: float, the sig det for the image
    :param seuilcosmic: float, cosmic threshold (cut off)
    
    :return: numpy array (1D), the extracted pixels for this order
    """
\end{lstlisting}

% -----------------------------------------------------------------------------------
\clearpage
\newpage
\section{The spirouFITS module}
% -----------------------------------------------------------------------------------

Contains functions to read FITS file and copy keywords

\subsection{read\_data}
\begin{lstlisting}[style=pythonstyle]
@read_data@(fitsfilename)
    """
    Reads "fitsfilename" using pyfits and returns array containing the data
    only uses data from the first header

    :param fitsfilename: string, file location and file name

    :return data: numpy array (2D), the image
    :return nx: int, the shape in the first dimension, i.e. data.shape[0]
    :return ny: int, the shape in the second dimension, i.e. data.shape[1]
    """
\end{lstlisting}

\noindent Note: currently not used?

\subsection{read\_data\_raw}
\begin{lstlisting}[style=pythonstyle]
@read_data_raw@(fitsfilename)
    """
    Reads "fitsfilename" using pyfits and returns array containing the data

    If there is one extension it is used
    If there are two extensions the second is used

    :param fitsfilename: string, file location and file name

    :return data: numpy array (2D), the image
    :return nx: int, the shape in the first dimension, i.e. data.shape[0]
    :return ny: int, the shape in the second dimension, i.e. data.shape[1]
    """
\end{lstlisting}

\noindent Note: Essentially the same as `read\_data' function

\subsection{read\_data\_all}
\begin{lstlisting}[style=pythonstyle]
@read_data_all@(fitsfilename)
    """
    Reads "fitsfilename" using pyfits and returns array containing the data

    If there is one extension it is used
    If there are two extensions the second is used
    If there are multiple extensions (1 or 2 extensions) the are joined in form:

    x = [[3, 3, 3],
         [3, 3, 3],
         [3, 3, 3]])

    y = [[5, 5, 5],
         [5, 5, 5],
         [5, 5, 5]])

    returned array is = [[3, 3, 3, 5, 5, 5],
                         [3, 3, 3, 5, 5, 5],
                         [3, 3, 3, 5, 5, 5]])

    :param fitsfilename: string, file location and file name

    :return data: numpy array (2D), the image
    :return nx: int, the shape in the first dimension, i.e. data.shape[0]
    :return ny: int, the shape in the second dimension, i.e. data.shape[1]
    """

\end{lstlisting}

\vspace{0.5cm}
\subsection{write\_data}
\begin{lstlisting}[style=pythonstyle]
@write_data@(fitsfilename, data)
    """
    Create the FITS file and write data

    :param filename: string, filename to save the fits file to
    :param data: numpy array (2D), the image

    :return None:
    """
\end{lstlisting}

\vspace{0.5cm}
\subsection{read\_ext}
\begin{lstlisting}[style=pythonstyle]
@read_ext@(fitsfilename)
    """
    Opens the FITS file and reads the number of extensions

    :param fitsfilename: string, filename to read
    :return length: the length of the extention -1
    """
\end{lstlisting}

\vspace{0.5cm}
\subsection{read\_key}
\begin{lstlisting}[style=pythonstyle]
@read_key@(fitsfilename, keyname, hdu=0)
    """
    Reads the "keyname" from the header of "fitsfilename" and returns
    the value of the "keyname"

    :param fitsfilename: string, filename to read
    :param keyname: string, the keyword to look for in the header,
                    returns 'UNKNOWN' if not found
    :param hdu: int, the extention to read the header from (default = 0)

    :return None:
    """
\end{lstlisting}

\vspace{0.5cm}
\subsection{read\_keys}
\begin{lstlisting}[style=pythonstyle]
@read_keys@(fitsfilename, keylist, hdu=0)
    """
    Reads the "keyname" from the header of "fitsfilename" and returns
    the value of the "keyname"

    :param fitsfilename: string, filename to read
    :param keylist: string, the keyword to look for in the header,
                    returns 'UNKNOWN' if not found
    :param hdu: int, the extention to read the header from (default = 0)

    :return None:
\end{lstlisting}

\noindent Note: currently identical to `read\_key'

\subsection{write\_newkey}
\begin{lstlisting}[style=pythonstyle]
@write_newkey@(fitsfilename, key)
    """
    Writes a new keyword "key[0]" to the FITS file header (extension = 0)
    :param fitsfilename: string, filename to write to
    :param key: list of string, the key to write, must be in form:

            key = [KEYWORD NAME, VALUE, COMMENT]

    :return None:
    """
\end{lstlisting}

\vspace{0.5cm}
\subsection{update\_key}
\begin{lstlisting}[style=pythonstyle]
@update_key@(fitsfilename, key)
    """
    Modifies a keyword "key[0]" in the FITS file header (extension = 0)
    
    :param fitsfilename: string, filename to read/write to
    :param key: string, the key to write, must be in form:

            key = [KEYWORD NAME, VALUE, COMMENT]

    :return None :
    """
\end{lstlisting}

\vspace{0.5cm}
\subsection{delete\_key}
\begin{lstlisting}[style=pythonstyle]
@delete_key@(fitsfilename, keyname)
    """
    Deletes a keyword "keyname" in the FITS file header (extension = 0)

    :param fitsfilename: string, filename to read/write to
    :param keyname: string, the key to write, must be in form:

    :return None:
    """
\end{lstlisting}

\vspace{0.5cm}
\subsection{writekey2dlist}
\begin{lstlisting}[style=pythonstyle]
@writekey2dlist@(fitsfilename, a, kw)
    """
    Writes a 2D list of values (in "a") to FITs file "fitsfilename" with
    keyword = "kw" + number

    where number = (row number * number of columns) + column number

    :param fitsfilename: string, filename to read/write to
    :param a: 2D list of values, the values to write
    :param kw: string, the prefix of the key to write, key will be in form:
                "kw" + number

    :return None:
    """
\end{lstlisting}

\vspace{0.5cm}
\subsection{copy\_key}
\begin{lstlisting}[style=pythonstyle]
@copy_key@(fits1, fits2)
    """
    Writes the header keys from one FITS fits (fits1) to another (fits2)
    
    Does not include FITS keywords:
    
            'SIMPLE', 'BITPIX', 'NAXIS', 'NAXIS1', 'NAXIS2',
            'EXTEND', 'COMMENT', 'CRVAL1', 'CRPIX1', 'CDELT1',
            'CRVAL2', 'CRPIX2', 'CDELT2', 'BSCALE', 'BZERO',
            'PHOT_IM', 'FRAC_OBJ', 'FRAC_SKY', 'FRAC_BB'
    
    :param fits1: string, filename to read from 
    :param fits2: string, filename to write to
    
    :return None: 
    """
\end{lstlisting}

\vspace{0.5cm}
\subsection{copy\_keys\_root}
\begin{lstlisting}[style=pythonstyle]
@copy_keys_root@(fits1, fits2, root)
    """
    Copies all keywords with prefix "root" from "fits1" to "fits2"
    
    :param fits1: string, filename to read from
    :param fits2: string, filename to write to
    :param root: string, prefix - start of the keyword to read/write to
    
    :return None: 
    """
\end{lstlisting}

\vspace{0.5cm}
\subsection{readkeyloco}
\begin{lstlisting}[style=pythonstyle]
@readkeyloco@(fitsfilename, kw, nbo, nbc)
    """
    Read a key from generated from a 2D list

    :param fitsfilename: string, filename to read/write to
    :param kw: string, the prefix of the key to write, key will be in form:
                "kw" + number
                
                where number = (row number * number of columns) + column number
                
    :param nbo: int, the row number
    :param nbc: int, the columns number
    
    :return None:
    """
\end{lstlisting}

% -----------------------------------------------------------------------------------
\clearpage
\newpage
\section{The spirouLOCOR module}
% -----------------------------------------------------------------------------------

Contains functions for order localization

\vspace{0.5cm}
\subsection{meas\_back}
\begin{lstlisting}[style=pythonstyle]
@meas_back@(y, backini, niter=3, coeff=9)
    """
    Measure and subtract the background from a column, iterative background fit
    by a polynomial and cut off high values
    
    :param y: 1D numpy array, column to use for background measurement
    :param backini: float, cut level
    :param niter: int, number of fits to perform
    :param coeff: int, polynomial order to fit
    :return yc: 1D numpy array, the image column with background subtracted
    :return f: 1D numpy array, the background fit data
    """
\end{lstlisting}

\noindent Note: Currently not used?
\noindent Note: required FORTRAN `fitpoly' function

\vspace{0.5cm}
\subsection{meas\_top}
\begin{lstlisting}[style=pythonstyle]
@meas_top@(yc, nbslice, nbpix)
    """
    Measure and normalize the top level of a column, seach for maximum on
    "nbslice" slices of "nbpix" pixels. Should consist of 1 or 2 orders. A 
    polynomial fit of order 9 is then divided through the original column

    :param yc: 1D numpy array, the image column
    :param nbslice: int, the number slices to use?
    :param nbpix: int, the number of pixels to use?
    
    :return ycc: 1D numpy array, the image column normalised by the fit
    :return f: 1D numpy array, the background fit data
    """
\end{lstlisting}

\noindent Note: Not sure exactly what this function is doing
\noindent Note: Currently not used?
\noindent Note: required FORTRAN `fitpoly' function

\vspace{0.5cm}
\subsection{meas\_minmax}
\begin{lstlisting}[style=pythonstyle]
@meas_minmax@(y, size)
    """
    Measure the minimum and maximum pixel value for each row using a box which
    contains all pixels for rows:  row-size to row+size and all columns.

    Edge pixels (0-->size and (image.shape[0]-size)-->image.shape[0] are
    set to the values for row=size and row=(image.shape[0]-size)

    :param y: numpy array (2D), the image
    :param size: int, the half size of the box to use (half height)
                 so box is defined from  row-size to row+size

    :return min_image: numpy array (1D length = image.shape[0]), the row values
                       for minimum pixel defined by a box of row-size to
                       row+size for all columns
    :retrun max_image: numpy array (1D length = image.shape[0]), the row values
                       for maximum pixel defined by a box of row-size to
                       row+size for all columns
    """
\end{lstlisting}

\vspace{0.5cm}
\subsection{meas\_min}
\begin{lstlisting}[style=pythonstyle]
@meas_min@(yc, nbslice, nbpix)
    """
    Measure and subtract the minimum level of a column, search for minimum on
    "nbslice" slices of "nbpix" pixels. Should consist of 1 or 2 orders. A
    polynomial fit of order 9 is then subtracted from the original column

    :param yc: 1D numpy array, the image column
    :param nbslice: int, the number slices to use?
    :param nbpix: int, the number of pixels to use?

    :return ycc: 1D numpy array, the image column subtracted by the fit
    :return f: 1D numpy array, the background fit data
    """
\end{lstlisting}

\noindent Note: Not sure exactly what this function is doing
\noindent Note: Currently not used?
\noindent Note: required FORTRAN `fitpoly' function

\vspace{0.5cm}
\subsection{poscolc}
\begin{lstlisting}[style=pythonstyle]
@poscolc@(ycc, seuil)
    """
    Takes the central pixel values and finds the order centers by looking for
    the start and end above threshold

    :param ycc: numpy array (1D) size = number of rows,
                the central pixel values
    :param seuil: float, the threshold above which to find pixels as being
                  part of an order

    :return positions: numpy array (1D), size= number of rows,
                       the pixel positions in cvalues where the centers of each
                       order should be
    """
\end{lstlisting}

\vspace{0.5cm}
\subsection{findord}
\begin{lstlisting}[style=pythonstyle]
@findord@(ycc, seuil, widthmin=5)
    """
    Takes a set of pixel values and finds the order center and width by 
    looking for the start and end above threshold "seuil"

    :param ycc: numpy array (1D) size = number of rows,
                an array of pixel values (across one order)
    :param seuil: float, the threshold above which to find pixels as being
                  part of an order
    :param widthmin" int, the minimum width of an order to be a valid, accepted
                     order

    :return position: float, the central pixel positions in ycc where the 
                      centers of this order should be
    :return width: float, the width in pixels of this order
    """
\end{lstlisting}

\vspace{0.5cm}
\subsection{locgaus}
\begin{lstlisting}[style=pythonstyle]
@locgaus@(y, ind1, ind2, sigdet, opt=1)
    """
    Find the position and the FWHM of one order using a gaussian using
    weights

    Non-Linear Fit of a Gaussian with 4 Parameters by the Method
    from Levenberg-Marquardt. Uses the subroutines MRQMIN, MRQCOF,
    COVSRT and GAUSSJ.

    :param y: 1D numpy array, the pixel values between ind1 and ind2
    :param ind1: float, the start point of this order
    :param ind2: float, the end point of this order
    :param sigdet: float, the ccdsigdet for the image
    :param opt: int, whether to use weights
                if opt=1 weights are used in the form:
                    weights = 1/(sqrt(y + sigdet^2))
                if opt=2 weights are used in the form:
                    weights = y 
                    
    :return center: float, the center of the order from the fit 
                    (set to zero if difference in y is less than 20*sigdet)
    :return fity: 1D numpy array, size=image size, the gaussian fit for y
    :return sigma: float, the FWHM of the order from the fit
                    (set to zero if difference in y is less than 20*sigdet)
    """
\end{lstlisting}

\noindent Note: Currently not used?
\noindent Note: required FORTRAN `fitgaus' function

\vspace{0.5cm}
\subsection{loc2gaus}
\begin{lstlisting}[style=pythonstyle]
@loc2gaus@(y, ind1, ind2, delta, sigdet, opt=1)
    """
    Find the position and the FWHM of one order using a two gaussian model with
    fixed widths (delta) using weights

    Non-Linear Fit of a Gaussian with 4 Parameters by the Method
    from Levenberg-Marquardt. Uses the subroutines MRQMIN, MRQCOF,
    COVSRT and GAUSSJ.

    :param y: 1D numpy array, the pixel values between ind1 and ind2
    :param ind1: float, the start point of this order
    :param ind2: float, the end point of this order
    :param delta: int
    :param sigdet: float, the ccdsigdet for the image
    :param opt: int, whether to use weights
                if opt=1 weights are used in the form:
                    weights = 1/(sqrt(y + sigdet^2))
                if opt=2 weights are used in the form:
                    weights = y

    :return center: float, the center of the order from the fit
                    (set to zero if difference in y is less than 20*sigdet)
    :return fity: 1D numpy array, size=image size, the gaussian fit for y
    :return fwhm: float, the FWHM of the order from the fit
                    (set to zero if difference in y is less than 20*sigdet)
    :return profile_error: float, the standard deviation of the residuals
                           between fit and data
    """
\end{lstlisting}

\noindent Note: Currently not used?
\noindent Note: required FORTRAN `fit2gaus' function

\vspace{0.5cm}
\subsection{fitprofil}
\begin{lstlisting}[style=pythonstyle]
@fitprofil@(y, sigdet)
    """
    Determine the profile of one order using a Gaussian model 
    (weights = 1/sqrt((y + sigdet^2)/max(y)))
    
    :param y: 1D numpy array, the pixel values of one order
    :param sigdet: float, the ccdsigdet for the image
    
    :return fw: float, the FWHM?
    :return fity: 1D numpy array, the gaussian fit to the data
    """
\end{lstlisting}

\noindent Note: Currently not used?
\noindent Note: required FORTRAN `fitgaus' function

\vspace{0.5cm}
\subsection{imaloco}
\begin{lstlisting}[style=pythonstyle]
@imaloco@(data, ac, fitsfilename)
    """
    Creates and save the image with the superposistion of the center of orders
    (set to zero)

    :param data: numpy array (2D), the image
    :param ac: numpy array (2D), size=(number of orders x polynomial fit level)
               the coefficient matrix (coefficients for all orders
    :param fitsfilename: string, file location and file name to save the file
    
    :return None:
    """
\end{lstlisting}

\noindent Note: Currently not used?

\vspace{0.5cm}
\subsection{imaloco2}
\begin{lstlisting}[style=pythonstyle]
@imaloco2@(data, ac, fitsfilename)
    """
    Creates and save the image with the superposistion of the center of orders
    (set to zero)

    :param data: numpy array (2D), the image
    :param ac: numpy array (2D), size=(number of orders x polynomial fit level)
               the coefficient matrix (coefficients for all orders
    :param fitsfilename: string, file location and file name to save the file
    
    :return None:
    """
\end{lstlisting}

\noindent Note: Exactly the same as imaloco?

\vspace{0.5cm}
\subsection{tableloco}
\begin{lstlisting}[style=pythonstyle]
@tableloco@(tblfilename, a)
    """
    Update the table localisation file

    Not current used in DRS

    :param tblfilename: string, filename of localisation file
    :param a: numpy array (2D), the localisation parameters for each order
              size = number of orders by number of localisation parameters

    :return None:
    """
\end{lstlisting}

\vspace{0.5cm}
\subsection{keyloco}
\begin{lstlisting}[style=pythonstyle]
@keyloco@(fitsfilename, ac, ass)
    """
    Update the localization keywords (CTR and SIG) on the fits
    file "fitsfilename".
    Keys are send to spirouFITS.write_listnewkey for writing

    Not current used in DRS

    :param fitsfilename: string, FITS file name
    :param ac: numpy array (2D), the position center fit coefficients
               size = number of orders x number of fit coefficients
    :param ass: numpy array (2D), the width fit coefficients
               size = number of orders x number of fit coefficients

    :return None:
    """
\end{lstlisting}

\vspace{0.5cm}
\subsection{keyloco2}
\begin{lstlisting}[style=pythonstyle]
@keyloco2@(fitsfilename, ac, ass)
    """
    Update the localization keywords (CTR and SIG) on the fits
    file "fitsfilename".
    Keys are written using pcfitsio (required module to use)

    Not current used in DRS

    :param fitsfilename: string, FITS file name
    :param ac: numpy array (2D), the position center fit coefficients
               size = number of orders x number of fit coefficients
    :param ass: numpy array (2D), the width fit coefficients
               size = number of orders x number of fit coefficients

    :return None:
    """
\end{lstlisting}

\vspace{0.5cm}
\subsection{writekeyloco3}
\begin{lstlisting}[style=pythonstyle]
@writekeyloco3@(fitsfilename, a)
    """
    Update the localization keywords on the fits file "fitsfilename".
    Keys are send to spirouFITS.write_listnewkey for writing

    Not current used in DRS

    :param fitsfilename: string, FITS file name
    :param a: numpy array (2D), the position center fit coefficients
               size = number of orders x number of fit coefficients

    :return None:
    """
\end{lstlisting}

\vspace{0.5cm}
\subsection{writekeyloco}
\begin{lstlisting}[style=pythonstyle]
@writekeyloco@(fitsfilename, a, kw)
    """
    Update the localization keywords (generic 2D 
    localisation coefficients) on the fits file "fitsfilename".
    Keys are send to spirouFITS.write_listnewkey for writing

    Not current used in DRS

    :param fitsfilename: string, FITS file name
    :param a: numpy array (2D), the "kw" fit coefficients
               size = number of orders x number of fit coefficients

    :return None:
    """
\end{lstlisting}

\vspace{0.5cm}
\subsection{readkeyloco}
\begin{lstlisting}[style=pythonstyle]
@readkeyloco@(fitsfilename, ic_locnbmaxo, ic_locdfitc)
    """
    Read 2D key list from "fitsfilename" header for key prefixes
    "CTR" and "SIG" size = (ic_locnbmaxo x ic_locdfitc)
    Keys are read using pcfitsio (required module to use)
    
    Not current used in DRS
    
    :param fitsfilename: string, FITS file name
    :param ic_locnbmaxo: int, number of orders to extract from header
    :param ic_locdfitc: int, number of fit coefficients to extract per order
    
    :return value: list, values from header size ic_locnbmaxo x ic_locdfitc
    """
\end{lstlisting}

\vspace{0.5cm}
\subsection{readkeyloco3}
\begin{lstlisting}[style=pythonstyle]
@readkeyloco3@(fitsfilename, rootkeyname)
    """
    Read 2D key list from "fitsfilename" header for key prefixes
    "rootkeyname" (up to the first 1000)
    Keys are read using pcfitsio (required module to use)

    Not current used in DRS

    :param fitsfilename: string, FITS file name
    :param rootkeyname: string, the header prefix to extract

    :return value: list, values from header
    """
\end{lstlisting}

\vspace{0.5cm}
\subsection{write\_fitsloco}
\begin{lstlisting}[style=pythonstyle]
@write_fitsloco@(fitsfilename, a, dim)
    """
    Write the fits localisation file
    
    :param fitsfilename: string, file name to write fits file to 
    :param a: numpy array (2D), the coefficient array
              size = number of orders x number of fit coefficients
    :param dim: int, the number of columns (x-axis dimension) 
    
    :return None: 
    """
\end{lstlisting}


\clearpage
\newpage
\section{The spirouRV module}
% -----------------------------------------------------------------------------------

\subsection{deltavrms2d}
\begin{lstlisting}[style=pythonstyle]
@deltavrms2d@(spe, wave, sigdet, seuilmax=350000., size=12):
    """
    Compute photon noise uncertainty
    
    :param spe: numpy array (2D), the extracted orders
    :param wave: numpy array (2D), the wave image
    :param sigdet: float, the noise or sigdet of the image
    :param seuilmax: float, the maximum threshold for good (unsaturated pixels)
    :param size: int, the size in pixels around a saturated pixel to not use
    
    :return dvrms2: numpy array (1D), photon noise uncertainty for each order
    :return meanpond: numpy array (1D), the weighted mean for each order
    """
\end{lstlisting}

\vspace{0.5cm}
\subsection{calcvrdrifts2d}

\begin{lstlisting}[style=pythonstyle]
@calcvrdrifts2d@(speref, spe, wave, sigdet, seuilmax=350000., size=12)
    """
    Compute the radial velocity drift between the reference extraction and the
    current extraction
    
    :param speref: numpy array (2D), the reference extracted orders
    :param spe: numpy array (2D), the current extracted orders
    :param wave: numpy array (2D), the wave image
    :param sigdet: float, the noise or sigdet of the image
    :param seuilmax: float, the maximum threshold for good (unsaturated pixels)
    :param size: int, the size in pixels around a saturated pixel to not use
    
    :return vrdrift: numpy array (1D) the drift for each order 
    """
\end{lstlisting}

\vspace{0.5cm}
\subsection{renorm\_cosmics2d}

\begin{lstlisting}[style=pythonstyle]
@renorm_cosmics2d@(speref, spe, seuilmax=350000., size=12, cut=4.5)
    """
    Correction of the cosmics and renormalisation by comparison with the 
    reference spectrum
    
    :param speref: numpy array (2D), the reference extracted orders
    :param spe: numpy array (2D), the current extracted orders
    :param seuilmax: float, the maximum threshold for good (unsaturated pixels)
    :param size: int, the size in pixels around a saturated pixel to not use
    :param cut: float, the number of sigma above to cut at (cosmic rejection)
    :return:
    """
\end{lstlisting}
