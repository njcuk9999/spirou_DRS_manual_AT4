\chapter{Using the DRS}


% -------------------------------------------------------------------------------------------
\section{Running the code}
\label{section:run_code}
% -------------------------------------------------------------------------------------------

To run the DRS python (assuming env\_setup is activated, see Section \ref{section:activate}) type:
\begin{lstlisting}[style=bashstyle]
DRS_{INSTRUMENT} -m
\end{lstlisting}  

\noindent i.e. 
\begin{lstlisting}[style=bashstyle]
DRS_spirou -m {PROGRAM NAME} {FOLDER} {FILES}
DRS_spirou -m cal_DARK_spirou {YYMMDD} {Filenames*}
\end{lstlisting}  

\noindent instead of python

\noindent Note: location should of script should be \textcolor{blue}{\{DIR\}/\{INSTALL\_ROOT\}/bin/DRS\{INSTRUMENT\}}

\noindent i.e. for `cal\_DARK\_spirou' in the \textcolor{blue}{\{DATA\_RAW\_ROOT\}/YYMMDD} directory one would result in something like the following:
\begin{lstlisting}[style=bashstyle]
DRS_spirou -m cal_DARK_spirou YYMMDD Filenames*
\end{lstlisting}
\begin{lstlisting}[style=bashstyle]
19:44:52.5 -   || *****************************************
19:44:52.5 -   || * SPIROU (\@) Geneva Observatory ()
19:44:52.5 -   || *****************************************
19:44:52.5 -   ||(dir_data_raw)      DRS_DATA_RAW=/data/spirou/drs/data/raw/
19:44:52.5 -   ||(dir_data_reduc)    DRS_DATA_REDUC=/data/spirou/drs/data/reduced/
19:44:52.5 -   ||(dir_drs_config)    DRS_CONFIG=/data/spirou/drs/INTROOT/DRS_SPIROU/config/
19:44:52.5 -   ||(dir_calib_db)      DRS_CALIB_DB=/data/spirou/drs/data/calibDB
19:44:52.5 -   ||(dir_data_msg)      DRS_DATA_MSG=/data/spirou/drs/data/msg/
19:44:52.5 -   ||(print_log)         DRS_LOG=1         %(0: minimum stdin-out logs)
19:44:52.5 -   ||(plot_graph)        DRS_PLOT=NONE            %(def/undef/trigger)
19:44:52.5 -   ||(used_date)         DRS_USED_DATE=undefined
19:44:52.5 -   ||(working_dir)       DRS_DATA_WORKING=/data/spirou/drs/data/tmp/
19:44:52.5 -   ||                    DRS_INTERACTIVE is not set, running on-line mode
19:44:52.5 -   |-c:+[...]|Now running : -c on file(s):  dark_dark...
19:44:52.5 -   |-c:+[...]|On directory /data/spirou/drs/data/raw/20170811
19:44:52.5 -   |-c:+[...]|ICDP loaded from: /data/spirou/drs/INTROOT/DRS_SPIROU/config/hadmrICDP_SPIROU.py
19:44:52.5 - * |-c:+[...]|Now processing Image TYPE DARK with -c recipe
19:44:52.5 -   |-c:+[...]|Reading Image /data/spirou/drs/data/raw/20170811/dark_dark02d.fits
19:44:52.5 -   |-c:+[...]|Image 2048x2048 loaded

\end{lstlisting}

% -------------------------------------------------------------------------------------------
\newpage
\section{Working example of the code for SPIRou}
\label{section:working_example}
% -------------------------------------------------------------------------------------------

\subsection{Overview}

For this example all files are from:
\begin{lstlisting}[style=bashstyle]
spirou\@10.102.14.81:/data/RawImages/H2RG-AT4/AT4-04/2017-07-10_15-36-18/ramps/
\end{lstlisting}  

Will also need current WAVE file from here:
\begin{lstlisting}[style=bashstyle]
spirou\@10.102.14.81:/data/reduced/DATA-CALIB/spirou_wave_ini3.fits
\end{lstlisting}  

\noindent This assumes a correct setup using all variables as in previous sections i.e.:
\begin{lstlisting}[style=text]
<INSTRUMENT_NAME>="SPIROU"
<DIR>="/data/spirou/drs"
<INSTALL_ROOT>="@$DIR@/INTROOT"
<DATA_ROOT>="@$DIR@/data"
<DATA_RAW_ROOT>="@$DATA_ROOT@/raw/"
<DATA_ROOT_REDUCED>="@$DATA_ROOT@/reduced/"
<DATA_ROOT_CALIB>="@$DATA_ROOT@/calibDB"
<DATA_ROOT_MSG>="@$DATA_ROOT@/msg/"
<DATA_ROOT_TMP>="@$DATA_ROOT@/tmp/"
<PYTHON_VERSION>="2.7"
<PYTHON_DIR>="@$DIR@/python/miniconda2/"
<GSL_DIR>="@$DIR@/c-libraries/gsl"
\end{lstlisting}

\noindent All files from 
\begin{lstlisting}[style=bashstyle]
spirou\@10.102.14.81:/data/RawImages/H2RG-AT4/AT4-04/2017-07-10_15-36-18/ramps/
\end{lstlisting}  
\noindent must be placed in a folder named `20170710' in \definevariable{DATA\_RAW\_ROOT} (\textcolor{blue}{/data/spirou/drs/data/raw}). \\

\noindent Starting with RAMP files and ending with extracted orders and calculated drifts we need to run six codes:
\begin{enumerate}
\item cal\_DARK\_spirou.py \hfill (See Section \ref{section:cal_DARK_spirou})
\item cal\_loc\_RAW\_spirou.py ($\times$2) \hfill (See Section \ref{section:cal_loc_RAW_spirou})
\item cal\_SLIT\_spirou.py \hfill (See Section \ref{section:cal_SLIT_spirou})
\item cal\_FF\_RAW\_spirou.py ($\times$2) \hfill (See Section \ref{section:cal_FF_RAW_spirou})
\item (add spirou\_wave\_ini3.fits to calibDB) 
\item cal\_extract\_RAW\_spirouAB.py and cal\_extract\_RAW\_spirouC.py (many times) \hfill (See Section \ref{section:cal_extract_RAW_spirou})
\item cal\_DRIFT\_RAW\_spirou \hfill (See Section \ref{section:cal_DRIFT_RAW_spirou})
\end{enumerate}

\subsection{Command-by-command run through}

\noindent We next list the exact commands used to go from the RAMPs to the extract orders and calculated drifts.
\begin{enumerate}
\item Change to the \textcolor{blue}{\{DIR\}} directory
\begin{lstlisting}[style=bashstyle]
@cd@ DIR
\end{lstlisting}  

\item Activate the environment (in bash only if not using the `.bashrc' method)
\begin{lstlisting}[style=bashstyle]
@source @env_setup.sh
\end{lstlisting}  

\item Check that environment is active (only if using `env\_setup.sh' and if not using the `.bashrc' method).
\begin{lstlisting}[style=bashstyle]
@echo @$DRS_ACTIVE
\end{lstlisting}  

\noindent result should be = 1

\item run the dark extraction on the `dark\_dark' file:
\begin{lstlisting}[style=bashstyle]
@DRS_spirou -m cal_DARK_spirou.py 20170710 @dark_dark02d406.fits
\end{lstlisting}  

\item run the order localisation on the `dark\_flat' files:
\begin{lstlisting}[style=bashstyle]
@DRS_spirou -m cal_loc_RAW_spirou.py 20170710 @dark_flat02f10.fits dark_flat03f10.fits dark_flat04f10.fits dark_flat05f10.fits dark_flat06f10.fits
\end{lstlisting}  

\item run the order localisation on the `flat\_dark' files:
\begin{lstlisting}[style=bashstyle]
@DRS_spirou -m cal_loc_RAW_spirou.py 20170710 @flat_dark02f10.fits flat_dark03f10.fits flat_dark04f10.fits flat_dark05f10.fits flat_dark06f10.fits
\end{lstlisting}  

\item run the slit calibration on the `fp\_fp' files.
\begin{lstlisting}[style=bashstyle]
@DRS_spirou -m cal_SLIT_spirou.py 20170710 @fp_fp02a203.fits fp_fp03a203.fits fp_fp04a203.fits
\end{lstlisting}  

\item run the flat field creation on the `dark\_flat' files:

\noindent \textcolor{red}{Note: if using same files as above you will get an error message when running the file}.
\noindent To solve this open the `master\_calib\_SPIROU.txt' file located in \textcolor{blue}{\{DATA\_ROOT\_CALIB\}}. Edit the unix date in the line that begins `TILT' so that it is less than the unix date on rwos `ORDER\_PROFIL\_AB' (i.e. change it from 1499707515.0 to 1499705515.0).

\noindent i.e. the `master\_calib\_SPIROU.txt' file should look go from
\begin{lstlisting}[style=text]
DARK 20170710 dark_dark02d406.fits 07/10/17/16:37:48 1499704668.0
ORDER_PROFIL_C 20170710 dark_flat02f10_order_profil_C.fits 07/10/17/17:03:50 1499706230.0
LOC_C 20170710 dark_flat02f10_loco_C.fits 07/10/17/17:03:50 1499706230.0
ORDER_PROFIL_AB 20170710 flat_dark02f10_order_profil_AB.fits 07/10/17/17:07:08 1499706428.0
LOC_AB 20170710 flat_dark02f10_loco_AB.fits 07/10/17/17:07:08 1499706428.0
TILT 20170710 fp_fp02a203_tilt.fits 07/10/17/17:25:15 @1499707515.0@
\end{lstlisting}

\noindent to this:

\definecolor{shadecolor}{named}{Tan} 
\begin{lstlisting}[style=text]
DARK 20170710 dark_dark02d406.fits 07/10/17/16:37:48 1499704668.0
ORDER_PROFIL_C 20170710 dark_flat02f10_order_profil_C.fits 07/10/17/17:03:50 1499706230.0
LOC_C 20170710 dark_flat02f10_loco_C.fits 07/10/17/17:03:50 1499706230.0
ORDER_PROFIL_AB 20170710 flat_dark02f10_order_profil_AB.fits 07/10/17/17:07:08 1499706428.0
LOC_AB 20170710 flat_dark02f10_loco_AB.fits 07/10/17/17:07:08 1499706428.0
TILT 20170710 fp_fp02a203_tilt.fits 07/10/17/17:25:15 @1499705515.0@

\end{lstlisting}

\begin{lstlisting}[style=bashstyle]
@DRS_spirou -m cal_FF_RAW_spirou.py 20170710 @dark_flat02f10.fits dark_flat03f10.fits dark_flat04f10.fits dark_flat05f10.fits dark_flat06f10.fits
\end{lstlisting}  

\item run the flat field creation on the `flat\_dark' files:
\begin{lstlisting}[style=bashstyle]
@DRS_spirou -m cal_FF_RAW_spirou.py 20170710 @flat_dark02f10.fits flat_dark03f10.fits flat_dark04f10.fits flat_dark05f10.fits flat_dark06f10.fits
\end{lstlisting}  

\item Currently we do not create a new wavelength calibration file for this run. Therefore we need to get one from here:
\begin{lstlisting}[style=bashstyle]
spirou\@10.102.14.81:/data/reduced/DATA-CALIB/spirou_wave_ini3.fits
\end{lstlisting}  

\noindent then place it in the \textcolor{blue}{\{DATA\_ROOT\_CALIB\}} folder. You will also need to edit the `master\_calib\_SPIROU.txt' file located in \textcolor{blue}{\{DATA\_ROOT\_CALIB\}}. Add the folloing line to `master\_calib\_SPIROU.txt'
\begin{lstlisting}[style=text]
WAVE 20170710 spirou_wave_ini3.fits 07/10/17/17:03:50 1499706230.0
\end{lstlisting}

\noindent and the `master\_calib\_SPIROU.txt' should look like this:
\begin{lstlisting}[style=text]
DARK 20170710 dark_dark02d406.fits 07/10/17/16:37:48 1499704668.0
ORDER_PROFIL_C 20170710 dark_flat02f10_order_profil_C.fits 07/10/17/17:03:50 1499706230.0
LOC_C 20170710 dark_flat02f10_loco_C.fits 07/10/17/17:03:50 1499706230.0
ORDER_PROFIL_AB 20170710 flat_dark02f10_order_profil_AB.fits 07/10/17/17:07:08 1499706428.0
LOC_AB 20170710 flat_dark02f10_loco_AB.fits 07/10/17/17:07:08 1499706428.0
TILT 20170710 fp_fp02a203_tilt.fits 07/10/17/17:25:15 1499705515.0
FLAT_C 20170710 dark_flat02f10_flat_C.fits 07/10/17/17:03:50 1499706230.0
@WAVE 20170710 spirou_wave_ini3.fits 07/10/17/17:03:50 1499706230.0@
\end{lstlisting}

\item run the extraction files on the `hcone\_dark', `dark\_hcone', `hcone\_hcone', `dark\_dark\_AHC1', `hctwo\_dark', `dark\_hctwo', `hctwo-hctwo', `dark\_dark\_AHC2' and `fp\_fp'  files:
\begin{lstlisting}[style=bashstyle]
@DRS_spirou -m cal_extract_RAW_spirouAB.py 20170710 @hcone_dark02c61.fits  hcone_dark03c61.fits  hcone_dark04c61.fits  hcone_dark05c61.fits  hcone_dark06c61.fits
\end{lstlisting}  
\begin{lstlisting}[style=bashstyle]
@DRS_spirou -m cal_extract_RAW_spirouC.py 20170710 @dark_hcone02c61.fits  dark_hcone03c61.fits  dark_hcone04c61.fits  dark_hcone05c61.fits  dark_hcone06c61.fits
\end{lstlisting}  
\begin{lstlisting}[style=bashstyle]
@DRS_spirou -m cal_extract_RAW_spirouAB.py 20170710 @hcone_hcone02c61.fits  hcone_hcone03c61.fits  hcone_hcone04c61.fits  hcone_hcone05c61.fits  hcone_hcone06c61.fits
\end{lstlisting}  
\begin{lstlisting}[style=bashstyle]
@DRS_spirou -m cal_extract_RAW_spirouC.py 20170710 @hcone_hcone02c61.fits  hcone_hcone03c61.fits  hcone_hcone04c61.fits  hcone_hcone05c61.fits  hcone_hcone06c61.fits
\end{lstlisting}  
\begin{lstlisting}[style=bashstyle]
@DRS_spirou -m cal_extract_RAW_spirouAB.py 20170710 @dark_dark_AHC102d61.fits  dark_dark_AHC103d61.fits  dark_dark_AHC104d61.fits  dark_dark_AHC105d61.fits  dark_dark_AHC106d61.fits
\end{lstlisting}  
\begin{lstlisting}[style=bashstyle]
@DRS_spirou -m cal_extract_RAW_spirouC.py 20170710 @dark_dark_AHC102d61.fits  dark_dark_AHC103d61.fits  dark_dark_AHC104d61.fits  dark_dark_AHC105d61.fits  dark_dark_AHC106d61.fits
\end{lstlisting}  
\begin{lstlisting}[style=bashstyle]
@DRS_spirou -m cal_extract_RAW_spirouAB.py 20170710 @hctwo_dark02c61.fits  hctwo_dark03c61.fits  hctwo_dark04c61.fits  hctwo_dark05c61.fits  hctwo_dark06c61.fits
\end{lstlisting}  
\begin{lstlisting}[style=bashstyle]
@DRS_spirou -m cal_extract_RAW_spirouC.py 20170710 @hctwo_dark02c61.fits  hctwo_dark03c61.fits  hctwo_dark04c61.fits  hctwo_dark05c61.fits  hctwo_dark06c61.fits
\end{lstlisting}  
\begin{lstlisting}[style=bashstyle]
@DRS_spirou -m cal_extract_RAW_spirouAB.py 20170710 @dark_hctwo02c61.fits  dark_hctwo03c61.fits  dark_hctwo04c61.fits  dark_hctwo05c61.fits  dark_hctwo06c61.fits
\end{lstlisting}  
\begin{lstlisting}[style=bashstyle]
@DRS_spirou -m cal_extract_RAW_spirouC.py 20170710 @dark_hctwo02c61.fits  dark_hctwo03c61.fits  dark_hctwo04c61.fits  dark_hctwo05c61.fits  dark_hctwo06c61.fits
\end{lstlisting}  
\begin{lstlisting}[style=bashstyle]
@DRS_spirou -m cal_extract_RAW_spirouAB.py 20170710 @hctwo-hctwo02c61.fits  hctwo-hctwo03c61.fits  hctwo-hctwo04c61.fits  hctwo-hctwo05c61.fits  hctwo-hctwo06c61.fits
\end{lstlisting}  
\begin{lstlisting}[style=bashstyle]
@DRS_spirou -m cal_extract_RAW_spirouC.py 20170710 @hctwo-hctwo02c61.fits  hctwo-hctwo03c61.fits  hctwo-hctwo04c61.fits  hctwo-hctwo05c61.fits  hctwo-hctwo06c61.fits
\end{lstlisting}  
\begin{lstlisting}[style=bashstyle]
@DRS_spirou -m cal_extract_RAW_spirouAB.py 20170710 @dark_dark_AHC202d61.fits  dark_dark_AHC203d61.fits  dark_dark_AHC204d61.fits  dark_dark_AHC205d61.fits  dark_dark_AHC206d61.fits
\end{lstlisting}  
\begin{lstlisting}[style=bashstyle]
@DRS_spirou -m cal_extract_RAW_spirouC.py 20170710 @dark_dark_AHC202d61.fits  dark_dark_AHC203d61.fits  dark_dark_AHC204d61.fits  dark_dark_AHC205d61.fits  dark_dark_AHC206d61.fits
\end{lstlisting}  
\begin{lstlisting}[style=bashstyle]
@DRS_spirou -m cal_extract_RAW_spirouAB.py 20170710 @fp_fp02a203.fits fp_fp03a203.fits fp_fp04a203.fits
\end{lstlisting}  
\begin{lstlisting}[style=bashstyle]
@DRS_spirou -m cal_extract_RAW_spirouC.py 20170710 @fp_fp02a203.fits fp_fp03a203.fits fp_fp04a203.fits
\end{lstlisting}  

\item run the drift calculation on the `fp\_fp' files:
\begin{lstlisting}[style=bashstyle]
@DRS_spirou -m cal_DRIFT_RAW_spirou.py 20170710 @fp_fp02a203.fits fp_fp03a203.fits fp_fp04a203.fits
\end{lstlisting}  


\end{enumerate}
